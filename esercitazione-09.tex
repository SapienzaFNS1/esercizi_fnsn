\begin{Exercise}[title={Isospin}]
  Sapendo che il mesone $f_2$ \`e uno stato di spin $S=2$ e isospin
  $\ket{I,I_3}=\ket{0,0}$, calcolare il rapporto tra le sezioni
  d'urto:
  \[
  R = \frac{\sigma(f_2\to\pi^+\pi^-)}{\sigma(f_2\to\pi^0\pi^0)}
  \]
  assumendo che nel decadimento forte la simmetria di isospin sia
  perfettamente conservata.
\end{Exercise}
\begin{Answer}
  Possiamo esprimere la parte di isospin della funzione d'onda della $f_2$ come somma degli
  autostati di isospin dei pioni:
  \beqn
  \ket{\pi^+} = \ket{1,1} \\
  \ket{\pi^0} = \ket{1,0} \\
  \ket{\pi^-} = \ket{1,-1}
  \eeqn
  utilizzando i coefficienti di Clebsh-Gordan si ottiene:
  \[
  \ket{0,0} = \sqrt{\frac{1}{3}}\ket{1,1}\ket{1,-1} - \sqrt{\frac{1}{3}}\ket{1,0}\ket{1,0}
  + \sqrt{\frac{1}{3}}\ket{1,-1}\ket{1,1}
  \]
  Quindi sono possibili i due decadimenti in $\ket{\pi^+,\pi^-}$ e $\ket{\pi^-,\pi^+}$.
  Chiamando $A(\pi^+,\pi^-)$ e $A(\pi^-,\pi^+)$ le due rispettive ampiezze (elementi di matrice),
  l'ampiezza di transizione totale, comprendendo la probabilit\`a di transizione di isospin \`e:
  \beqn
  \bra{\pi^+,\pi^-}\ket{f_2} \propto \sqrt{\frac{1}{3}}A(\pi^+,\pi^-) \\
  \bra{\pi^-,\pi^+}\ket{f_2} \propto \sqrt{\frac{1}{3}}A(\pi^-,\pi^+)  
  \eeqn

  Il mesone $f_2$ ha spin $S=2$ (e quindi momento angolare totale
  $J=2$), quindi le due ampiezze $A(\pi^+,\pi^-)$ e $A(\pi^-,\pi^+)$
  sono legate tra loro dalla simmetria di parit\`a:
  \[
  A(\pi^+,\pi^-) = (-1)^2 A(\pi^+,\pi^-) = A(\pi^+,\pi^-)
  \]
  e quindi:
  \[
  \sigma(f_2 \to \pi^+\pi^-) \propto \left| 2 \cdot \sqrt{\frac{1}{3}}A(\pi^+,\pi^-) \right|^2
  \]

  La parte di isospin dell'ampiezza di transizione in due pioni neutri \`e data da:
  \[
  \bra{\pi^0,\pi^0}\ket{f_2} \propto \sqrt{\frac{1}{3}}A(\pi^0,\pi^0)
  \]

  Poich\'e lo stato finale \`e costituito da due bosoni ($S=0$) identici, la funzione d'onda va
  simmetrizzata correttamente, tenendo conto del teorema di spin-orbita.
  Ad esempio, per due particelle:
  \begin{equation}
    \label{eqn:bose2part}
    A(1,2) \rightarrow \frac{A(1,2)+A(2,1)}{\sqrt{2}}
  \end{equation}
  o, in generale, tenendo conto che le ampiezze $A$ non dipendono dall'ordine delle particelle, e
  quindi sono tutte uguali tra loro, si ha, per $n$ particelle identiche:
  \begin{equation}
    \label{eqn:boseNpart}
    A(1,2,\dots,n) \rightarrow \frac{n!A(1,2,\dots,n)}{\sqrt{n!}}    
  \end{equation}

  Quindi, nel caso dei due pioni neutri, usando la Eq.~\ref{eqn:boseNpart}:
  \[
  A(\pi^0,\pi^0) \rightarrow \frac{2 \cdot A(\pi^0,\pi^0)}{\sqrt{2}}
  \]
  Assumendo la simmetria di isospin esatta, l'elemento di matrice del decadimento ($A$) in pioni
  di qualsiasi carica \`e lo stesso, e lo \`e anche lo spazio delle fasi. Quindi il rapporto delle sezioni d'urto vale:
  \beqn
  R &=& \frac{\sigma(f_2\to\pi^+\pi^-)}{\sigma(f_2\to\pi^0\pi^0)} =
  \frac{\left|\bra{\pi^+\pi^+}\ket{f_2}\right|^2}{\left|\bra{\pi^0\pi^0}\ket{f_2}\right|^2} \\
  &=&\frac{\left| 2\cdot \sqrt{\frac{1}{3}}\right|^2}{\left| \sqrt{2}\cdot \sqrt{\frac{1}{3}}\right|^2} = 2
  \eeqn
\end{Answer}

\begin{Exercise}[title={Raggio di nuclei simmetrici}]
  I nulcei di \ce{^{27}_{14}Si} e \ce{^{27}_{13}Al} sono simmetrici,
  cio\`e i loro stati fondamentali sono identici, eccetto per la loro
  carica elettrica.

  \Question Conoscendo la loro differenza di massa ($\Delta
  M=\SI{6}{MeV}$), stimare il loro raggio.  Per semplicit\`a, si
  trascuri la differenza di massa tra protone e neutrone.

\end{Exercise}
\begin{Answer}
  Usando la stima dell'energia elettrostatica accumulata in una sfera uniformemente carica di raggio $R$,
  approssimazione che solitamente usiamo per la distribuzione di una carica $Q$ all'interno di un nucleo:
  \[
  U = \frac{3}{5}\cdot\frac{Q^2}{R}
  \]
  le ipotesi del problema (trascurare la differenza di massa
  potone-neutrone) ci dicono che possiamo attribuire la differenza di
  massa tra i due nuclei alla sola differenza di carica elettrostatica.
  Quindi:
  \[
  \Delta U = \frac{3e^2}{5R}(Z^2_1 - Z^2_2)
  \]
  Usando i dati numerici per i nuclei in gioco:
  \beqn
  R &=& \frac{3e^2}{5\Delta U}\times\left(Z_1^2-Z_2^2\right) = \frac{3\hbar c}{5\Delta U}\times\left( \frac{e^2}{\hbar c} \right)\times \left( 14^2 - 13^2 \right) \\
  &=& \frac{3\times \SI{197}{MeV\cdot fm}}{5\times\SI{6}{MeV}} \times \frac{1}{137}\times \left( 14^2 - 13^2 \right) \\
  &=& \SI{3.88}{fm}
  \eeqn
\end{Answer}

\begin{Exercise}[title={Differenza di massa tra nuclei di un multipletto di isospin}]
  I nuclei di \ce{^{90}_{40}Zr} e \ce{^{90}_{39}Y} appartengono allo stesso multipletto di isospin.
  \Question Si stimi la differenza di massa tra i due nuclei.
\end{Exercise}
\begin{Answer}
  La differenza di massa tra due membri dello stesso multipletto di
  isospin \`e dovuta solamente alla differenza di energia
  elettrostatica (termine Coulombiano della formula
  \textit{semi-empirica} delle masse di Bethe-Weizs{\"a}cker) e dalla differenza
  di massa tra il neutrone e il protone.

  Quindi, chiamando $Z$ il numero atomico del nucleo pi\`u leggero
  (\ce{^{90}_{39}Y}), e con $U$ l'energia elettrostatica di un nucleo,
  si ha che la differenza di massa cercata \`e uguale alla differnza
  di energie dei due nuclei:
  \beqn
  \Delta M &=& E(A,Z+1)-E(A,Z) = \Delta U - (m_n - m_p)c^2 \\
  &=& \frac{3e^2}{5R}(2Z-1) - \SI{0.780}{MeV} = \frac{3(2Z+1)\hbar c}{5R}\times\frac{1}{137} -  \SI{0.780}{MeV}
  \eeqn
  Usiamo come stima del raggio nucleare:
  \[
  R\approx1.2\times A^{1/3}\SI{}{fm} = 1.2 \times 90^{1/3}\SI{}{fm} \approx \SI{5.38}{fm} 
  \]
  da cui:
  \beqn
  \Delta M &=& \frac{3(2\times 39\times 39+1)\times \SI{197}{MeV\cdot fm}}{5\times\SI{5.38}{fm}\times 137} - \SI{0.780}{MeV} \\
  &=& \SI{11.89}{MeV}
  \eeqn
\end{Answer}

\begin{Exercise}[title={Densit\`a di massa nucleare}]
  Stimare la densit\`a di massa nucleare (in \SI{}{g/cm^3}), approssimando il nucleo come
  una sfera con massa distribuita uniformemente e trascurando la
  differenza di massa tra protone e neutrone.
\end{Exercise}
\begin{Answer}
  Se si approssima la massa del neutrone a quella del protone: $m_n=m_p$, e il raggio del nucleo come:
  \[
  R = R_0 \times A^{1/3} \approx \SI{1.2}{MeV\cdot fm} \times A^{1/3}
  \]
  che \`e una buona approssimazione a grandi valori di $A$, si ottiene, per la densit\`a di una sfera
  con massa distriuita uniformemente:
  \beqn
  \rho &=& \frac{M}{V} = \frac{A\cdot m_p}{4/3 \pi R_0^3 A} = \frac{3 m_p}{4\pi R_0^3} \\
  &=& \frac{3\times\SI{1.67e-24}{g}}{12.56\times (\SI{1.2e-13}{cm})^3} \approx \SI{2.3e14}{g/cm^3}
  \eeqn
\end{Answer}

\begin{Exercise}[title={Catena di decadimenti radioattivi ed equilibrio secolare}]
  Si consideri la seguente catena di decadimenti radioattivi:
  \[
  \ce{^{244}Pu} \to \ce{^{240}U} \to \ce{^{240}Np} \to \ce{^{240}Pu}
  \]
  I tempi di dimezzamento dei nuclei che decadono sono $t^{1/2}(\ce{^{244}Pu})=\SI{81e6}{anni}$,
  $t^{1/2}(\ce{^{240}U})=\SI{14}{ore}$ e $t^{1/2}(\ce{^{240}Np})=\SI{67}{min}$. Se, in un certo momento, si ha a disposizione
  una massa corrispondente a \SI{1}{grammoatomo} di \ce{^{244}Pu} puro, si determini:

  \Question quanti nuclei di \ce{^{240}U} e  \ce{^{240}Np} saranno presenti dopo 1 mese;
  \Question qual \`e il tipo di radioattivit\`a di ciascun decadimento;
  \Question qual \`e l'attivit\`a del primo decadimento.
  
\end{Exercise}

\begin{Answer}
  \begin{enumerate}
  \item Detti $\tau_1$ e $N_1$ vita media e numero di \ce{^{244}Pu} al
    tempo $t$, $\tau_2$ e $N_2$ vita media e numero di \ce{^{240}U} e
    $\tau_3$ e $N_3$ vita media e numero di \ce{^{240}Np}, si ha
    $\tau_1 \gg \tau_2 , \tau_3$. Inoltre, se $t$ \`e il tempo della
    misura ($t=\SI{30}{giorni}$), si ha anche $t \gg \tau_2 , \tau_3$.

    In queste condizioni, vale la relazione dell'\textit{equilibrio secolare}:
    \[
    \frac{N_1}{\tau_1} \approx \frac{N_2}{\tau_2} \approx \frac{N_3}{\tau_3}
    \]
    Poich\'e $N_1=N_A$, si ha, per il numero di nuclei di \ce{^{240}U}:
    \beqn
    N(\ce{^{240}U}) &=& N_2 = \frac{\tau_2}{\tau_1}\cdot N_1 = \frac{t^{(1/2)}_2}{t^{(1/2)}_1}\cdot N_A \\
    &\approx& \frac{\SI{14}{ore}}{\SI{8.1e7}{} \times 365 \times \SI{24}{ore}} \cdot \SI{6e23}{} \\
    &\approx& \SI{1.2e13}{}
    \eeqn
    analogamente per i nulcei di \ce{^{240}Np}:
    \beqn
    N(\ce{^{240}Np}) &=& N_3 = \frac{\tau_3}{\tau_1}\cdot N_1 = \frac{t^{(1/2)}_3}{t^{(1/2)}_1}\cdot N_A \\
    &\approx& \frac{\SI{67}{min}}{\SI{8.1e7}{} \times 365 \times 24 \times \SI{60}{min}} \cdot \SI{6e23}{} \\
    &\approx& \SI{9.4e11}{}
    \eeqn

  \item    I decadimenti in oggetto sono:
    \beqn
    \ce{^{244}Pu} &\rightarrow& \ce{^{240}U} + \alpha \\
    \ce{^{240}U}  &\rightarrow& \ce{^{240}Np} + e^- + \bar{\nu}_e \\
    \ce{^{240}Np} &\rightarrow& \ce{^{240}Pu} + e^- + \bar{\nu}_e
    \eeqn
    
  \item L'attivit\`a misurata nel decadimento $\alpha$ sar\`a:
    \beqn
    A &=& \left| \dv{N_1}{t} \right| \approx \frac{N_1}{\tau_1} = \frac{N_1 \ln 2}{t^{(1/2)}_1} \\
    &\approx& \frac{6e23\times 0.69}{\SI{8.1e7}{} \times 365 \times 24 \times \SI{3600}{s}} \approx \SI{1.6e8}{s^{-1}} = \SI{160}{MHz}
    \eeqn
    
  \end{enumerate}
\end{Answer}

\begin{Exercise}[title={Decadimento $\alpha$}]
  Nello spettro del decadimento $\alpha$ del $\ce{^{240}Pu}\to\ce{^{236}U}+\alpha$ si osservano due righe, corrispondenti alle
  energie cinetiche delle particelle $\alpha$ a \SI{5.17}{MeV} e \SI{5.12}{MeV}.
  \Question Si determinino i Q-valori dei due decadimenti.
  \Question La riga meno energetica corrisponde al decadimento in un livello eccitato \ce{^{236}U^*} che si diseccita al livello
  fondamentale con emissione $\gamma$. Determinare l'energia dei $\gamma$ emessi.
\end{Exercise}
\begin{Answer}
  \begin{enumerate}
  \item  Chiamiamo $X$ il nucleo genitore che decade $\alpha$ e $Y$ il nucleo
    figlio.  Nel sistema di riferimento di $X$ la conservazione di
    energia e impulso si scrive:
    \[
    \begin{cases}
      M_X c^2 = M_Y c^2 + M_X c^2 + K_Y + K_\alpha \\
      \vec p_X + \vec p_Y = 0
    \end{cases}
    \]
    dove abbiamo indicato con $K_Y$ e $K_\alpha$ le energie cinetiche
    del nucleo figlio e della particella $\alpha$. Dalla conservazione
    dell'impulso si ha $\vert\vec p_X\vert = \vert\vec p_Y\vert \equiv
    p$. Dalla conservazione dell'energia:
    \beqn
    (M_X - M_Y - M_\alpha)c^2 \equiv Q &=& \frac{p^2}{2M_Y} + \frac{p^2}{2M_\alpha} = \frac{p^2}{2M_\alpha}\left[ 1+\frac{M_\alpha}{M_Y} \right] \\
    &=& K_\alpha\left[ 1+\frac{M_\alpha}{M_Y} \right]
    \eeqn
    Da questa relazione otteniamo l'energia cinetica della particella $\alpha$, usando l'espansione al
    prim'ordine in $\epsilon=M_\alpha/M_Y$, che vale per valori di $A$ grandi:
    \begin{equation}
      \label{eqn:QvalKalpha}
      K_\alpha = \frac{Q}{1+\frac{M_\alpha}{M_Y}} \approx Q\left(1-M_\alpha/M_Y\right) \approx Q\left(1-\frac{4}{A}\right)
    \end{equation}
    Con lo stesso approccio si ricaverebbe l'energia cinetica corrispondente al rinculo del nucleo figlio:
    \[
    Q = \frac{p^2}{2M_Y}\left[ 1+\frac{M_Y}{M_\alpha} \right] = K_Y \left[ 1 + \frac{1}{M_\alpha/M_Y}\right] = K_Y \frac{1+M_\alpha/M_Y}{M_\alpha/M_Y}
    \]
    da cui si ottiene:
    \begin{equation}
      \label{eqn:QvalKY}
      K_Y = Q  \frac{M_\alpha/M_Y}{1+M_\alpha/M_Y} \approx Q \frac{M_\alpha}{M_Y}\left[1-\frac{M_\alpha}{M_Y}\right] \approx Q\frac{M_\alpha}{M_Y} \approx Q\frac{4}{A}
    \end{equation}
    in cui abbiamo usato la stessa espensione al prim'ordine del
    deonominatore e tenuto solo il termine di ordine 1 in
    $\epsilon=M_\alpha/M_Y$ del prodotto. Da qui si vede che per $A$
    grande, $K_\alpha \gg K_Y$, cio\`e il rinculo del nucleo figlio \`e
    piccolo rispetto a quello dell'$\alpha$.

    Dall'eq.~\ref{eqn:QvalKalpha} si ottiene:
    \[
    Q \approx \frac{K_\alpha}{1-4/A} = \frac{A}{A-4} K_\alpha
    \]
    quindi numericamente:
    \beqn
    Q_1 \approx \frac{240}{236} \times \SI{5.17}{MeV} = \SI{5.26}{MeV} \\
    Q_2 \approx \frac{240}{236} \times \SI{5.12}{MeV} = \SI{5.21}{MeV}
    \eeqn

  \item L'energia del $\gamma$ emesso \`e uguale alla differenza dei
    due Q-valori, quindi tra $Q_1(\ce{^{240}Pu}\to\ce{^{236}U})$ e
    $Q_2(\ce{^{240}Pu}\to\ce{^{236}U^*})$, quindi:
    \[
    E_\gamma = Q_1 - Q_2 = \SI{0.05}{MeV} = \SI{50}{keV}
    \]
  \end{enumerate}
  
\end{Answer}


\begin{Exercise}[title={Decadimenti permessi e proibiti}]
Stabilire quali delle seguenti reazioni sono possibili dandone la  motivazione:
\begin{itemize}
\item{$K^{-} + p \to \Omega^{-} + K^{+} + K^{0} $}
\item{$\psi \to \pi^+ + \pi^0 + \pi^-$}
\item{$\pi^- + p \to \Sigma^+ + K^-$}
\item{$\pi^- + p \to \pi^0 + \pi^0 $}
\item{$p + p \to n + \Delta^{++} + p + \bar{p}$}
\end{itemize}
\end{Exercise}
\begin{Answer}
  Le reazioni in esame sono tutte dovute all'interazione forte pertanto si 
  ha la conservazione dei seguenti numeri quantici: carica elettrica $C$,
  numero barionico $B$, numero leptonico $L$, stranezza $S$ ed isospin $I_3$.
  Le reazioni proposte nell'esercizio sono rispettivamente:
  
  \begin{itemize}
  \item{$K^{-} + p \to \Omega^{-} + K^{+} + K^{0} $: possibile}
  \item{$\psi \to \pi^+ + \pi^0 + \pi^-$: possibile}
  \item{$\pi^- + p \to \Sigma^+ + K^-$: impossibile, viola la conservazione di $S$}
  \item{$\pi^- + p \to \pi^0 + \pi^0 $: impossibile, viola la conservazione di $B$}
  \item{$p + p \to n + \Delta^{++} + p + \bar{p}$: possibile}
  \end{itemize}

\end{Answer}

\begin{Exercise}[title={Simmetrie discrete in transizioni nucleari}]
L'energia di legame dei nuclei \ce{^4_2He} e \ce{^7_3Li} \`e \SI{28.3}{MeV} e
\SI{39.3}{MeV} rispettivamente. Si consideri la reazione $p + \ce{^7_3Li} \to\ce{^4_2He} + \ce{^4_2He}$.

\Question Verificare se la reazione \`e esotermica  ($Q>0$) o endotermica ($Q < 0$).

\Question Calcolare l'energia minima del protone perch\'e la reazione avvenga.

\Question Lo stato di spin-parit\`a del nucleo \ce{^7_3Li} \`e
$J^P=(3/2)^-$.  Calcolare il valore del momento angolare orbitale
iniziale, assumendo che quello finale sia nullo.
\end{Exercise}

\begin{Answer}
  \begin{enumerate}

    \item La reazione  $p + \ce{^7_3Li} \to\ce{^4_2He} + \ce{^4_2He}$ ha un Q-valore pari a:
      \beqn
      Q &=& M_p + M_{\ce{Li}} - 2 M_\alpha = M_p + [3 M_p + 4 M_n - B(\ce{^7_3Li})] - 2 [2 M_p + 2 M_n - B(\alpha)] = \\
      &=& 2 B(\alpha) - B(\ce{^7_3Li}) = 2 \cdot \SI{28.3}{MeV} - \SI{39.3}{MeV} = \SI{17.3}{MeV} > 0
      \eeqn
      La reazione \`e quindi esotermica.

      \item L'energia minima richiesta \`e dovuta alla barriera
        coulombiana. Assumendo il raggio del protone trascurabile in
        confronto a quello del \ce{^7_3Li} (che stimiamo come
        $R(\ce{^7_3Li})=R_0 A^{1/3} \approx \SI{1.2}{fm} A^{1/3}$), si
        ha:
        \[
        K_{min} = \frac{z Z e^2}{4 \pi \epsilon_0 d} =  \frac{z Z \alpha \hbar c}{R(\ce{^7_3Li})} = 
        \frac{3\cdot \SI{197}{MeV fm}}{137 \cdot \SI{1.2}{fm}\cdot 7^{1/3}} \approx \SI{1.9}{MeV}
        \]
        
        \item Indicando i valori di spin-parit\`a dei nuclei coinvolti nella reazione si ha:
          \[
          p\left(\frac{1}{2}^+\right) + \ce{^7_3Li}\left(\frac{3}{2}^-\right) \to \ce{^4_2He}(0^+) + \ce{^4_2He}(0^+)
          \]
          Sapendo che il momento angolare orbitale dello stato finale \`e nullo, tale deve essere anche
          il momento angolare totale finale. Pertanto la conservazione del momento angolare impone:
          \[
          \frac{1}{2} \oplus \frac{3}{2} \oplus L_i = 0
          \]
          avendo indicato con $\oplus$ l'operazione di addizione dei momenti angolari e con $L_i$
          il momento angolare orbitale iniziale. Poich\'e $\frac{1}{2} \oplus \frac{3}{2} = 1,2$, 
          ne segue che anche $L_i$ pu\`o assumere i valori 1 o 2.

          La conservazione della parit\`a, a sua volta, impone che la parit\`a dello stato iniziale sia 
          uguale a quella finale, cio\`e sia +1. Sapendo che la parit\`a del protone \`e +1, si ha allora: 
          \[
          P_i = P(p) \times P(\ce{^7_3Li}) \times  P_{orb} = (+1) \times (-1) \times (-1)^{L_i}
          \]
          da cui segue che $L_i$ deve essere dispari. Quindi risulta $L_i = 1$.
  \end{enumerate}
\end{Answer}
