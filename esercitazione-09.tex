\begin{Exercise}[title={Isospin}]
  Sapendo che il mesone $f_2$ \`e uno stato di spin $S=2$ e isospin
  $\ket{I,I_3}=\ket{0,0}$, calcolare il rapporto tra le sezioni
  d'urto:
  \[
  R = \frac{\sigma(f_2\to\pi^+\pi^-)}{\sigma(f_2\to\pi^0\pi^0)}
  \]
  assumendo che nel decadimento forte la simmetria di isospin sia
  perfettamente conservata.
\end{Exercise}
\begin{Answer}
  Possiamo esprimere la parte di isospin della funzione d'onda della $f_2$ come somma degli
  autostati di isospin dei pioni:
  \beqn
  \ket{\pi^+} = \ket{1,1} \\
  \ket{\pi^0} = \ket{1,0} \\
  \ket{\pi^-} = \ket{1,-1}
  \eeqn
  utilizzando i coefficienti di Clebsh-Gordan si ottiene:
  \[
  \ket{0,0} = \sqrt{\frac{1}{3}}\ket{1,1}\ket{1,-1} - \sqrt{\frac{1}{3}}\ket{1,0}\ket{1,0}
  + \sqrt{\frac{1}{3}}\ket{1,-1}\ket{1,1}
  \]
  Quindi sono possibili i due decadimenti in $\ket{\pi^+,\pi^-}$ e $\ket{\pi^-,\pi^+}$.
  Chiamando $A(\pi^+,\pi^-)$ e $A(\pi^-,\pi^+)$ le due rispettive ampiezze (elementi di matrice),
  l'ampiezza di transizione totale, comprendendo la probabilit\`a di transizione di isospin \`e:
  \beqn
  \bra{\pi^+,\pi^-}\ket{f_2} \propto \sqrt{\frac{1}{3}}A(\pi^+,\pi^-) \\
  \bra{\pi^-,\pi^+}\ket{f_2} \propto \sqrt{\frac{1}{3}}A(\pi^-,\pi^+)  
  \eeqn

  Il mesone $f_2$ ha spin $S=2$ (e quindi momento angolare totale
  $J=2$), quindi le due ampiezze $A(\pi^+,\pi^-)$ e $A(\pi^-,\pi^+)$
  sono legate tra loro dalla simmetria di parit\`a:
  \[
  A(\pi^+,\pi^-) = (-1)^2 A(\pi^+,\pi^-) = A(\pi^+,\pi^-)
  \]
  e quindi:
  \[
  \sigma(f_2 \to \pi^+\pi^-) \propto \left| 2 \cdot \sqrt{\frac{1}{3}}A(\pi^+,\pi^-) \right|^2
  \]

  La parte di isospin dell'ampiezza di transizione in due pioni neutri \`e data da:
  \[
  \bra{\pi^0,\pi^0}\ket{f_2} \propto \sqrt{\frac{1}{3}}A(\pi^0,\pi^0)
  \]

  Poich\'e lo stato finale \`e costituito da due bosoni ($S=0$) identici, la funzione d'onda va
  simmetrizzata correttamente, tenendo conto del teorema di spin-orbita.
  Ad esempio, per due particelle:
  \begin{equation}
    \label{eqn:bose2part}
    A(1,2) \rightarrow \frac{A(1,2)+A(2,1)}{\sqrt{2}}
  \end{equation}
  o, in generale, tenendo conto che le ampiezze $A$ non dipendono dall'ordine delle particelle, e
  quindi sono tutte uguali tra loro, si ha, per $n$ particelle identiche:
  \begin{equation}
    \label{eqn:boseNpart}
    A(1,2,\dots,n) \rightarrow \frac{n!A(1,2,\dots,n)}{\sqrt{n!}}    
  \end{equation}

  Quindi, nel caso dei due pioni neutri, usando la Eq.~\ref{eqn:boseNpart}:
  \[
  A(\pi^0,\pi^0) \rightarrow \frac{2 \cdot A(\pi^0,\pi^0)}{\sqrt{2}}
  \]
  Assumendo la simmetria di isospin esatta, l'elemento di matrice del decadimento ($A$) in pioni
  di qualsiasi carica \`e lo stesso, e lo \`e anche lo spazio delle fasi. Quindi il rapporto delle sezioni d'urto vale:
  \beqn
  R &=& \frac{\sigma(f_2\to\pi^+\pi^-)}{\sigma(f_2\to\pi^0\pi^0)} =
  \frac{\left|\bra{\pi^+\pi^+}\ket{f_2}\right|^2}{\left|\bra{\pi^0\pi^0}\ket{f_2}\right|^2} \\
  &=&\frac{\left| 2\cdot \sqrt{\frac{1}{3}}\right|^2}{\left| \sqrt{2}\cdot \sqrt{\frac{1}{3}}\right|^2} = 2
  \eeqn
\end{Answer}
