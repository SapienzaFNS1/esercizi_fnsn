\documentclass{article}
\usepackage[utf8]{inputenc}
\usepackage{url}
\usepackage{hyperref}
\usepackage{graphicx}
\usepackage{amsmath}
\usepackage{commath}
\usepackage[noanswer]{exercise}
%\usepackage{exercise}
\usepackage{siunitx}
\usepackage{physics} % derivate
\usepackage{lscape} % landscape
\usepackage{mhchem} % formule chimiche

\newcommand{\vett}[1]{\ensuremath{\textbf{#1}}}
\newcommand{\quadriv}[1]{\ensuremath{\underline{#1}}}

\newcommand{\beq}{\begin{equation*}}
\newcommand{\eeq}{\end{equation*}}
\newcommand{\beqn}{\begin{eqnarray*}}
\newcommand{\eeqn}{\end{eqnarray*}}

%%% TO RENAME THE EXERCISES IN ITALIAN
\renewcommand\listexercisename{Lista degli esercizi}%
\renewcommand\ExerciseName{Esercizio}%
\renewcommand\AnswerName{Soluzione dell'esercizio}%
\renewcommand\ExerciseListName{Es.}%
\renewcommand\AnswerListName{Soluzione}%
\renewcommand\ExePartName{Parte}%
%%%%%%%%%%%%%%%%%%

\pagestyle{empty}
%%%%%%%%%%%%%%%%%%%%%%%%%%%
\hoffset = -2.0 cm
\voffset = -3.0 cm
\textheight = 25.0 cm
\marginparwidth = 1.0 cm
\evensidemargin = 1.0 cm
\textwidth =16.1 cm
\renewcommand{\arraystretch}{1.5}
%%%%%%%%%%%%%%%%%%%%%%%%%%%

\title{Esercitazioni di Fisica Nucleare e Subnucleare 1}
\author{Emanuele Di Marco,\footnote{\href{mailto:emanuele.dimarco@roma1.infn.it}{emanuele.dimarco@roma1.infn.it}} \\
Valerio Ippolito,\footnote{\href{mailto:valerio.ippolito@roma1.infn.it}{valerio.ippolito@roma1.infn.it}} }
\date{\today}

\begin{document}

\maketitle

%%%%%%%%%%%%%%%%%%%%%%%%%
%%% ESERCITAZIONE 2
%%% FATTA IL 03/03/2021
%%%%%%%%%%%%%%%%%%%%%%%%

\begin{Exercise}[title={Applicazione delle trasformazioni di Lorentz}]
  Un osservatore $S$ unidimensionale, posizionato sull'asse \textit{x}, vede un lampo rosso a $x_R=\SI{1210}{m}$ e,
  dopo un tempo di 4.96\,$\mu s$, un lampo blu a $x_B=\SI{480}{m}$.
  \begin{enumerate}
    \item Qual \`e la velocit\`a di $S$ per un osservatore $S'$ che vede accadere gli eventi nello stesso punto?
    \item Quale evento avviene prima, secondo $S'$, e qual \`e
      l'intervallo temporale misurato da lui tra i due lampi di luce?
      (\textit{Suggerimento:} provare a calcolarlo usando le
      trasformazioni di Lorentz, oppure calcolarlo usando l'intervallo
      spazio-temporale.
  \end{enumerate}
\end{Exercise}
\begin{Answer}
  \begin{enumerate}
    \item L'osservatore $S'$ sta viaggiando a una velocit\`a $v$, tale che vede il lampo rosso (evento $R$) e
      poi arriva in $B$ esattamente nel momento in cui avviene il lampo blu. Quindi nel suo sistema di riferimento:
      \begin{align*}
        &x'_R = x'_B \\
        &x_R - vt_R = x_B - vt_B,
      \end{align*}
      e quindi
      \begin{equation*}
        v = \frac{x_R - x_B}{t_R-t_B} =
        \frac{\SI{1210}{m} - \SI{480}{m}}{0-\SI{4.96e-6}{s}} =
        \SI{-1.47e8}{m/s}
      \end{equation*}

    \item Sebbene la misura degli intervalli temporali possa cambiare
      nel passare da $S$ a $S'$, l'ordine degli eventi deve essere lo
      stesso in qualsiasi sistema di riferimento. Quindi l'osservatore in $S'$ vede l'evento R avvenire prima dell'evento B.

      Adesso troviamo l'intervallo temporale in $S'$, che si muove a
      velocit\`a $v=\SI{-1.47e8}{m/s}$ rispetto a $S$. Applicando le
      trasformazioni di Lorentz abbiamo:
      \begin{align*}
        t'_R &= \gamma\left(t_R-\frac{v}{c^2}x_R\right) = \frac{1}{\sqrt{1-\left(\frac{\SI{1.47e8}{m/s}}{\SI{3e8}{m/s}}\right)^2}}
        \left[0-\frac{\SI{-1.47e8}{m/s}}{(\SI{3e8}{m/s})^2}(\SI{1210}{m})\right]=\SI{2.27}{\mu s} \\
        t'_B &= \gamma\left(t_B-\frac{v}{c^2}x_B\right) = \frac{1}{\sqrt{1-\left(\frac{\SI{1.47e8}{m/s}}{\SI{3e8}{m/s}}\right)^2}}
        \left[\SI{4.96e-6}{s}-\frac{\SI{-1.47e8}{m/s}}{(\SI{3e8}{m/s})^2}(\SI{480}{m})\right]=\SI{6.59}{\mu s} \\
      \end{align*}
      Quindi l'evento $R$ accade prima dell'evento $B$, con una
      differenza temporale $\Delta r'= t'_B-t'_R=\SI{4.32}{\mu s}$.

      Possiamo anche calcolare $\Delta t'$ usando l'intervallo
      invariante dello spazio tempo, tenendo conto del fatto che in
      $S'$ i due eventi accadono nello stesso punto dello spazio
      (i.e. $\Delta x'=0$):
      \begin{align*}
        &(c\Delta t')^2-(\Delta x')^2 = (c\Delta t')^2 = (c\Delta t')^2 = (c\Delta t)^2-(\Delta x)^2 \\
        \Delta t' &= \sqrt{(\Delta t)^2-\left(\frac{\Delta x}{c}\right)^2} \\
        & = \sqrt{ \SI{4.96e-6}{s}^2 - \left( \frac{\SI{1210}{m}-\SI{480}{m}}{\SI{3e8}{m/s}} \right)^2} = \SI{4.32}{\mu s}
      \end{align*}
      ottenendo lo stesso valore che avevamo trovato con le trasformazioni di Lorentz.
  \end{enumerate}
\end{Answer}

\begin{Exercise}[title={Contrazione delle lunghezze}]
  Un osservatore misura la lunghezza di un'asta quando questa \`e a
  riposo, ottenendo $L=\SI{1}{m}$, e quando \`e in moto, ottenendo
  $L'=\SI{0.5}{m}$. A che velocit\`a viaggia l'asta quando \`e in moto?
\end{Exercise}
\begin{Answer}
La lunghezza a riposo \`e legata alla lunghezza misurata quando l'asta \`e
in movimento dalla relazione $L=L'/\gamma$, per cui $\gamma=2$. La
velocit\`a dell'asta \`e dunque
\begin{align*}
    \gamma &= \frac{1}{\sqrt{1-\beta^2}} = 2\\
    \frac{1}{2} &= \sqrt{1-\beta^2}\\
    \beta^2&=\frac{v^2}{c^2} = \frac{3}{4}\\
    v &= \frac{\sqrt{3}}{4}c=\SI{2.6e8}{m/s}.
\end{align*}
\end{Answer}


\begin{Exercise}[title={Decadimento e dilatazione dei tempi}]
Met\`a dei muoni di un fascio composto da muoni di energia fissata
sopravvive dopo aver viaggiato $l=\SI{600}{m}$ nel sistema di
riferimento del laboratorio. Qual è la velocità dei muoni, conoscendo la vita media del muone $\tau_0=\SI{2.2}{\mu s}$?
\end{Exercise}
\begin{Answer}
La legge di decadimento dei muoni \`e di tipo esponenziale, e quindi:
\begin{equation*}
    \frac{N}{N_0} = \exp\left(-\frac{t}{\tau}\right) = \exp\left(-\frac{vt}{v\tau}\right)=\exp\left(-\frac{l}{\beta c\gamma \tau_0}\right)=\frac{1}{2}
\end{equation*}
da cui segue che
\begin{align*}
    -\log\frac{1}{2}&=\frac{l}{\beta\gamma c\tau_0}\\
    \beta\gamma&=\frac{\beta}{\sqrt{1-\beta^2}} = -\frac{l}{\log\frac{1}{2}c\tau_0}\equiv\lambda,
\end{align*}
ed elevando al quadrato
\begin{equation*}
    \beta=\sqrt{\frac{\lambda^2}{1+\lambda^2}}\approx 0.80.
\end{equation*}
\end{Answer}


\begin{Exercise}[title={Composizione relativistica delle velocit\`a e contrazione delle lunghezze}]
  Due razzi, di lunghezza a riposo $L_0$, si avvicinano alla Terra da direzioni opposte, con velocit\`a $\pm c/2$. Quanto appare lungo un razzo all'altro razzo?
\end{Exercise}
\begin{Answer}
  Mettiamoci nel SR di uno dei razzi (razzo 1) e calcoliamo quanto
  viaggia velocemente l'altro (razzo 2) rispetto al SR del razzo 1.
  Il testo ci dice che nel sistema della Terra, il razzo 1 ha
  velocit\`a $c/2$ e il razzo 2 ha velocit\`a $-c/2$. Applicando la
  composizione delle velocit\`a, e indicando con l'apice il SR del razzo 1 nel quale vogliamo misurare la velocit\`a:
  \begin{equation*}
    v_2' = \frac{(v_2-v_1)}{1-v_1v_2/c^2} = \frac{(-c/2)-(c/2)}{1-(c/2)(-c/2)/c^2} = -\frac{4}{5}c,
  \end{equation*}
  quindi il razzo 2 appare come se si stia avvicinando a $0.8c$. Una volta nota la velocit\`a del razzo 2 nel SR del razzo 1, la
  contrazione delle lunghezze di Lorentz d\`a:
  \begin{equation*}
    L' = \frac{L_0}{\gamma} = L_0\sqrt{1-\left(\frac{4}{5}\right)^2} = \frac{3}{5}L_0.
  \end{equation*}
\end{Answer}

\begin{Exercise}[title={Il decadimento del muone, visto dal muone}]
  Il muone, indicato con $\mu$, \`e una particella instabile che
  decade con un tempo proprio (vita media per il muone a riposo)
  $\tau_0=\SI{2.2}{\mu s}$. Se viene prodotto all'inizio
  dell'atmosfera per la collisione di raggi cosmici energetici con
  particelle nelle molecole d'aria. Se assumiamo che i muoni vengano
  prodotti all'inizio dell'atmosfera tutti a un'altezza di
  \SI{10}{km}, e hanno una velocit\`a $v=0.999c$, in media i muoni
  raggiungono la superficie della Terra prima di decadere?
\end{Exercise}
\begin{Answer}
  In classe abbiamo svolto l'esercizio dal punto di vista (sistema di riferimento) della Terra.
  In quel caso il muone viaggia per una distanza media $d = v\gamma\tau_0$ prima di decadere, cio\`e:
  \begin{equation*}
    d = v\gamma\tau = \frac{(\SI{2.997e8}{m/s})(\SI{2.2e-6}{s})}{\sqrt{1-\left(\frac{\SI{2.997e8}{m/s}}{\SI{3e8}{m/s}}\right)^2}} = \SI{14.5}{km}
  \end{equation*}
  Quindi con un fattore:
  \begin{equation*}
    \gamma = \frac{1}{\sqrt{1-\left(\frac{\SI{2.997e8}{m/s}}{\SI{3e8}{m/s}}\right)^2}} \simeq 22
  \end{equation*}
  invece che la breve distanza $d'=v\tau_0=\SI{660}{m}$ che penseremmo, non tenendo conto della dilatazione dei tempi.

  E nel sistema di riferimento del muone? Nel SR solidale con il muone, \`e l'atmosfera a viaggiare con $v_{atm}=0.999c$ e quindi il suo spessore si
  contrae di un fattore $\gamma \ simeq 22$, e quindi la lunghezza misurata da lui \`e:  
  \begin{equation*}
    L' = \frac{L_{atm}}{\gamma} = \frac{\SI{15e3}{m}}{22}  = \SI{450}{m}
  \end{equation*}
  Il muone, che vive in media un tempo $\tau_0$, pu\`o volare per una
  distanza media pari a $c\tau_0=\SI{660}{m}$, che per lui \`e
  maggiore dello spessore dell'atmosfera, e quindi pu\`o raggiungere
  terra prima di decadere.
\end{Answer}
\end{document}

%%%%%%%%%%%%%%%%%%%%%%%%%
%%% ESERCITAZIONE 3
%%% FATTA IL 10/03/2021
%%%%%%%%%%%%%%%%%%%%%%%%

\begin{Exercise}[title={Energia cinetica}]
Quanto lavoro bisogna compiere per aumentare la velocità di un elettrone ($m=\SI{511}{keV/c^2}$) dalla posizione di riposo a:
\Question $0.50c$?
\Question $0.990c$?
\Question $0.9990c$?
\end{Exercise}
\begin{Answer}
A questi elettroni dovremo dare una certa energia cinetica $K$, in modo da far passare l'energia totale da quella a riposo ($\gamma=0$), cio\`e
\begin{equation*}
    E_i = mc^2,
\end{equation*}
a
\begin{equation*}
    E_f = K + mc^2. 
\end{equation*}
notare che la massa dell'elettrone $m$ \`e la massa a riposo.
Dalla relazione
\begin{equation*}
    E_f = m\gamma c^2
\end{equation*}
segue
\begin{equation*}
    K = E_f - mc^2 = m(\gamma - 1) c^2 = m\left(\frac{1}{\sqrt{1-\frac{v^2}{c^2}}}-1\right)c^2,
\end{equation*}
per cui nei tre casi indicati servono rispettivamente \SI{79}{keV}, \SI{3.1}{MeV} e \SI{10.9}{MeV}.
\end{Answer}


\begin{Exercise}[title={Energia cinetica}]
Si calcoli la velocità di una particella in modo che
\Question la sua energia cinetica sia il doppio della sua energia a riposo
\Question la sua energia totale sia il doppio della sua energia a riposo
\end{Exercise}
\begin{Answer}
Usiamo la relazione
\begin{equation*}
    E = m\gamma c^2 = K+mc^2 = m(\gamma -1)c^2 + mc^2,
\end{equation*}
assieme alla definizione
\begin{equation*}
    \gamma=\frac{1}{\sqrt{1-v^2/c^2}},
\end{equation*}
per cui si ha che
\begin{equation*}
    v = \sqrt{1-\frac{1}{\gamma^2}}c,
\end{equation*}
e quindi
\begin{itemize}
    \item $K=2mc^2 = m(\gamma-1) c^2\rightarrow \gamma = 3 \rightarrow v \approx  0.94c \approx \SI{2.8e8}{m/s}$;

    \item $E=2mc^2=m\gamma c^2\rightarrow \gamma = 2\rightarrow v = \SI{2.6e8}{m/s}$.
\end{itemize}
\end{Answer}


\begin{Exercise}[title={Dilatazione dei tempi e quadrimpulso}]
Nell’urto ad alta energia di una particella di radiazione cosmica con
un’altra particella nella parte alta dell’atmosfera terrestre,
\SI{120}{km} sopra il livello del mare, si genera un pione di energia
totale $E=\SI{1.35e5}{MeV}$ che si muove verticalmente verso il
basso. Nel sistema di riferimento ad esso solidale, il pione decade
dopo \SI{35}{ns} dalla sua creazione. A che altitudine sopra il
livello del mare, nel sistema di riferimento terrestre, avviene il
decadimento? L’energia a riposo di un pione è \SI{139.6}{MeV/c^2}.
\end{Exercise}
\begin{Answer}
Se nel suo sistema di riferimento il pione decade dopo
$\tau=\SI{35}{ns}$, nel sistema di riferimento del laboratorio questo
tempo si sarà dilatato, diventando $t=\tau\gamma$. In questo tempo
$t$, prima di decadere il pione avrà percorso una distanza
\begin{equation*}
    L = v t = (\beta c) (\gamma \tau).
\end{equation*}
L'espressione di $\gamma$ in funzione delle variabili note, $E$ ed
$m$, la ricaviamo da $E=m\gamma c^2$ segue che $\gamma=E/(mc^2)$. Per
ottenere quella di $\beta\gamma$, invece, osserviamo che dalla
relazione $p=m\gamma v = m\gamma\beta/c$ segue che $\beta\gamma =
pc/m$, e che da $(pc)^2+(mc^2)^2=E^2$, si ha che
\begin{equation*}
    \beta\gamma = pc/m = \frac{\sqrt{E^2-(mc^2)^2}}{m} = \sqrt{\gamma^2-1} = \sqrt{\left(\frac{E}{mc^2}\right)^2-1}.
\end{equation*}
Poiché la massa del pione è di \SI{139.6}{MeV/c}, questo decadrà nel laboratorio dopo aver percorso
\begin{equation*}
    L = \beta\gamma c\tau \approx
    \sqrt{\left(\frac{\SI{1.35e5}{MeV}}{\SI{139.6}{MeV/c^2}c^2}\right)^2-1}\cdot\SI{30}{cm/ns}\cdot\SI{35}{ns}
    \approx \SI{ 10.1}{km},
\end{equation*}
ovvero a $\SI{120}{km}-\SI{10.1}{km}\approx\SI{110}{km}$ sul livello del mare.
\end{Answer}


\begin{Exercise}[title={Conservazione dell'energia e impulso}]
  Un elettrone $e^-$ con energia cinetica \SI{1.0}{MeV} collide
  frontalmente su un positrone $e^+$ fermo (il positrone \`e
  l'anti-particella dell'elettrone, che ha la stessa massa, ma carica
  opposta). Nella collisione le due particelle si annichilano e il
  risultato della reazione sono due fotoni di uguale energia, ognuno
  dei quali viaggia a un angolo $\theta$ rispetto alla direzione del
  moto (il fotone \`e una particella di massa nulla, il quanto della
  radiazione elettromagnetica, con energia $E=pc$. La reazione \`e:
  \begin{equation*}
    e^- + e^+ \rightarrow 2\gamma
  \end{equation*}
\Question Calcolare l'energia $E$, l'impulso $p$ e l'angolo di
emissione $\theta$ di ciascun fotone.
\end{Exercise}

\begin{Answer}
  La massa a riposo dell'elettrone \`e $m=\SI{0.511}{MeV/c^2}$, e identifichiamo con la direzione positiva dell'asse $x$
  il suo vettore impulso $p$, e con $K$ la sua energia cinetica. L'energia totale del sistema si pu\`o ottenere come
  somma di energia cinetica e energia a riposo:
    \begin{equation*}
      E = K + 2mc^2 = \SI{2.022}{MeV}
    \end{equation*}
    Poich\'e i due fotoni hanno stessa massa (nulla, in particolare),
    essi emergono dalla collisione con la stessa energia. Applicando
    la conservazione dell'energia dallo stato iniziale allo stato finale:
    \begin{equation*}
      E_\gamma = \frac{1}{2}E = \SI{1.011}{MeV}
    \end{equation*}
    Per ottenere l'impulso, considerando che il fotone ha massa nulla:
    \begin{equation*}
      p_\gamma = \sqrt{\frac{E_\gamma^2}{c^2}-m_\gamma^2c^2} = E_\gamma/c = \SI{1.011}{MeV/c}
    \end{equation*}
    Per calcolare l'angolo $\theta$ con cui emergono i due fotoni (uno
    con $+\theta$ e uno con $\-theta$ rispetto alla direzione
    dell'asse $x$), applichiamo la conservazione dell'impulso lungo la direzione $x$:
    \begin{equation*}
      p = 2p_\gamma\cos\theta
    \end{equation*}
    Dobbiamo trovare $p$ del sistema, che \`e quello dell'elettrone incidente, poich\'e il positrone \`e fermo. Quindi:
    \begin{equation*}
      p^2c^2 = E^2 - m^2c^4  = (K+mc^2)^2 - m^2c^4 = K^2 + m^2c^4 2mKc^2 - m^2c^4 =
      K(K+2mc^2)
    \end{equation*}
    e quindi:
    \begin{equation*}
    p = \sqrt{K(K+2mc^2)}/c = \sqrt{\SI{1}{MeV}(\SI{1}{MeV}+2\SI{0.511}{MeV/c^2}c^2)}/c = \SI{1.422}{MeV/c}
    \end{equation*}
    Usando questo valore di $p$ si ottiene $\theta=45.3^\circ$.
\end{Answer}

\begin{Exercise}[title={Sistema di riferimento del laboratorio e del centro di massa }]
  Si consideri un fascio di antiprotoni di impulso \SI{0.65}{GeV/c}
  che impattano su un bersaglio di atomi di idrogeno.  In questa
  collisione, se c'\`e energia sufficiente, potrebbero prodursi dei
  barioni $\Lambda$, particelle contenenti un quark up ($u$), un quark
  down ($d$) e un terzo quark, che pu\`o strange ($s$), charm ($c$) o
  bottom ($b$). Quello con il quark $s$ ha una massa a riposo
  $m_\Lambda=m_{\bar\Lambda}=\SI{1.116}{GeV/c^2}$, dove $\bar\Lambda$
  \`e l'anti-particella del barione $\Lambda$.
  \Question La reazione $\bar p p\to\Lambda\bar\Lambda$ pu\`o avvenire?
\end{Exercise}
  
\begin{Answer}
  La risoluzione del problema \`e facilitata calcolando la massa invariante in due sistemi
  di riferimento diversi. Nello stato iniziale, i quadrimpulsi nel laboratorio sono:
  \begin{eqnarray*}
    P_1 &=& [E_1/c, \vec p_1] \\
    P_2 &=& [m_pc,\vec 0].
  \end{eqnarray*}
  La massa invariante \`e quindi:
  \begin{equation*}
    \sqrt{s} = \sqrt{2m_p^2c^4+2E_1m_pc^2}.
  \end{equation*}
  L'energia minima per far avvenire la reazione \`e quella per cui le
  particelle dello stato finale sono a riposo nel sistema di
  riferimento del centro di massa. In questo sistema di riferimento si ha:
  \begin{equation*}
    \sqrt{s} = 2m_\Lambda c^2.
  \end{equation*}
  Applicando la conservazione della massa invariante prima e dopo la reazione:
  \begin{equation*}
    4m_\Lambda^2c^4 = 2 m_p^2c^4 + 2 E_1m_pc^2
  \end{equation*}
  da cui si ottiene:
  \begin{equation*}
    E_1 = \frac{2m_\Lambda c^2-m_p^2c^2}{m_p} = \SI{1.72}{GeV}.
  \end{equation*}
  Da questo, l'impulso del fascio alla soglia deve essere:
  \begin{equation*}
    p_1c = \sqrt{E_1^2 - m_p^2 c^4} = \SI{1.4}{GeV}.
  \end{equation*}
  Quindi con un fascio di antiprotoni con impulso di soli
  \SI{0.65}{GeV/c} su targhetta non si possono produrre delle coppie
  $\Lambda-\bar\Lambda$.
  
\end{Answer}


%%%%%%%%%%%%%%%%%%%%%%%%%
%%% ESERCITAZIONE 4
%%% FATTA IL 19/03/2021
%%%%%%%%%%%%%%%%%%%%%%%%

\begin{Exercise}[title={Unit\`a di misura}]
Usando il fatto che $\hbar c = \SI{197.3}{MeV fm}$, si dimostri che in un sistema di unità di misura in cui $\hbar=c=1$ vale:
\Question $\SI{1}{GeV^{-2}}=\SI{0.389}{mb}$
\Question $\SI{1}{m}=\SI{5.068e15}{GeV^{-1}}$
\Question $\SI{1}{s}=\SI{1.5e24}{GeV^{-1}}$
\end{Exercise}
Ricordiamo che $\SI{1}{b} = \SI{1e-28}{m^2}$ e che 
\begin{equation*}
    [\hbar c]=[\si{J s} \si{m/s}] = [E][L].
\end{equation*}
\begin{Answer}
L'idea è di capire per quale potenza di $\hbar c$ e $c$ va moltiplicato il termine a sinistra di ciascuna equazione, per ottenere il termine di destra. 

Per cui:
\begin{itemize}
    \item $[ \SI{1}{GeV^{-2}} ] [\hbar c]^\alpha = [E]^{-2} [E] ^\alpha [L]^\alpha = [\SI{0.389}{mb}] = [L]^2$, da cui segue $\alpha=2$ e $\SI{1}{GeV^{-2}}(\hbar c)^2=\frac{\SI{197.3}{MeV fm}}{\SI{1}{GeV}} = \left(\frac{\SI{0.1973e-15}{GeV m}}{\SI{1}{GeV}}\right)^2=\SI{0.389}{mb}$;

    \item $[ \SI{1}{m} ] [\hbar c]^\alpha = [L] [E] ^\alpha [L]^\alpha = [\SI{5.068e15}{GeV^{-1}}] = [E]^{-1}$, da cui segue $\alpha=-1$ e $\SI{1}{m}(\hbar c)^{-1}=\frac{\SI{1}{m}}{\SI{197.3}{MeV fm}} = \frac{\SI{1}{m}}{\SI{0.1973e-15}{GeV m}}=\SI{5.068e15}{GeV^{-1}}$;
    
    \item $[ \SI{1}{s} ] [\hbar c]^\alpha[c]^\beta  = [T] [E] ^\alpha [L]^\alpha [L]^\beta [T]^{-\beta} = [T] [E]^\alpha [L]^{\alpha+\beta} [T]^{-\beta} = [\SI{1.5e24}{GeV^{-1}}] = [E]^{-1}$, da cui segue $\alpha=-1, \beta=1$ e $\SI{1}{s}(\hbar c)^{-1}c=\frac{\SI{1}{s}}{\SI{197.3}{MeV fm}}\SI{299 792 458}{m/s} 
    = \frac{\SI{299 792 458}{m}}{\SI{0.1973e-15}{GeV m}} = \SI{1.5e24}{GeV^{-1}}$.
\end{itemize}
\end{Answer}


\begin{Exercise}[title={Energia di soglia di una reazione}]
  Nell’urto protone-nucleo calcolare l’energia cinetica di soglia
  minima e massima per la reazione protone su protone di un nucleo di
  rame:
  \begin{equation*}
    p_1 + p_2 \to p + p + \pi^+ \pi^-
  \end{equation*}
  sapendo che il moto di Fermi del protone nel nucleo-bersaglio ha un
  impulso medio di $p_F = \SI{0.240}{GeV/c}$ e che la massa del pione
  carico \`e: $m(\pi^\pm =\SI{0.140}{GeV/c^2}$ e la massa del protone
  \`e $m_p= \SI{0.938}{GeV/c^2}$.
\end{Exercise}

\begin{Answer}
L'impulso di Fermi regola il moto dei nucleoni (protoni e neutroni)
all'interno dei nuclei. Se lo trascuriamo l’energia cinetica di soglia
per la reazione è pari a:
\begin{equation*}
K_1 = \frac{(2m_p + 2m_\pi)^2-(2m_p)^2}{2m_p} = 4m_\pi + \frac{2m_\pi^2}{m_p} = \SI{0.602}{GeV}
\end{equation*}
L'energia totale \`e la somma di energia cinetica e della massa della particella prodotta:
\begin{equation*}
E_1 = K_1 + m_p = \SI{1.540}{GeV}
\end{equation*}

Considerando invece il moto del protone legato nel nucleo di rame,
moto diretto casualmente rispetto alla direzione del protone
incidente, si ha l’energia di soglia minima (massima) quando l’impulso
di Fermi del protone del nucleo è antiparallelo (parallelo) alla
direzione del protone incidente.

Conviene calcolare la massa invariante nel sistema del laboratorio:
\begin{equation*}
  \sqrt{s} = \sqrt{(E_1+E_2)^2-\abs{\vec p_1+\vec p_2}^2} = \sqrt{(E_1+E_2)^2-(p_1+p_F)^2} =
  \sqrt{2m_p^2 + 2E_1E_2 \pm 2p_1p_F}
\end{equation*}
l'energia del protone nel nucleo \`e $E_2 = \sqrt{m_p^2+m_F^2}=\SI{968}{MeV}$.
L'energia di soglia nel centro di massa \`e quando tutte le particelle dello stato finale sono ferme. Quindi \`e:
\begin{equation*}
\sqrt{s}= 2m_p+2m_\pi
\end{equation*}
Usando l'uguaglianza della massa invariante nello stato iniziale e finale:
\begin{eqnarray*}
  E_1E_2\pm p_1p_F = 2(m_p+m_\pi)^2-m_p^2 \\
  E_1E_2 - 2(m_p+m_\pi)^2 + m_p^2 = \pm p_F\sqrt{E_1^2-m_p^2}
\end{eqnarray*}
La parte $2(m_p+m_\pi)^2 + m_p^2$ \`e una costante, e vale $A=\SI{1.444}{GeV^2}$. Sostituendola:
\begin{eqnarray*}
  E_1E_2 - A =  \pm p_F\sqrt{E_1^2-m_p^2} \\
  p_F^2E_1^2 - p_F^2m_p^2 = E_1^2E_2^2 + A^2-2AE_1E_2 \\
  E_1^2(E_2^2-p_F^2) - 2AE_2E_1 + p_F^2m_p^2 + A^2=0
\end{eqnarray*}

e usando $E_2^2-p_F^2=m_p^2$ si trova:
\begin{eqnarray*}
  E_1^2m_p^2 - 2AE_1E_2 + p_F^2m_p^2+A^2 = 0 \\
  E_1^2-\frac{2AE_2}{m_p^2}E_1 + \frac{A^2}{m_p^2}+p_F^2=0
\end{eqnarray*}
Questa \`e un'equazione di secondo grado che ha soluzioni:
\begin{equation*}
  E_1 = \frac{AE_2}{m_p^2} \pm \sqrt{\left(\frac{AE_2}{m_p^2}\right)^2-\frac{A^2}{m_p^2}-p_F^2}
\end{equation*}  
Usando $E_2=\SI{968}{MeV}$ si ricavano i due valori di energia di soglia massimo e  minimo:
\begin{eqnarray*}
E_1^{max} \approx \SI{1.3}{GeV} \\
E_1^{min} \approx \SI{1.9}{GeV}
\end{eqnarray*}
Da cui si ricavano anche le energie cinetiche minime e massime:
\begin{eqnarray*}
K_1^{max} = E_1^{max} - m_p \approx \SI{1}{GeV} \\
K_1^{min} = E_1^{min} - m_p \approx \SI{0.3}{GeV} \\
\end{eqnarray*}

\end{Answer}



\begin{Exercise}[title={Energia di soglia di una reazione su bersaglio e collisore}]
  Calcolare l’energia di soglia per la reazione:
  \beq
  e^+e^- \to \mu^+\mu^-
  \eeq
  \Question su bersaglio fisso
  \Question in collisioni $e^+e^-$ con fasci di pari energia
\end{Exercise}
\begin{Answer}
  
Per la reazione su bersaglio fisso.
Come di frequente, usiamo due sistemi di riferimento diversi. Nello stato iniziale, considerando i quadrimpulsi nel laboratorio la massa invariante \`e:
\beq
\sqrt{s} = \sqrt{2m_e^2 + 2 E_1m_e}
\eeq
Nello stato finale, alla soglia (muoni fermi) si ha:
\beq
\sqrt{s} = 2m_\mu
\eeq
Uguagliando le due espressioni per la massa invariante si ottiene $E_1 = \SI{44}{GeV}$ alla soglia.

Per la reazione in collisioni $e^+e^-$ con fasci di pari energia, il sistema del centro di massa coincide con quello del laboratorio.
Nello stato iniziale:
\beq
\sqrt{s} = 2E_e
\eeq
Nello stato finale, alla soglia (ancora una volta, muoni fermi) si ha:
\beq
\sqrt{s} = 2m_\mu
\eeq
Uguagliando le due espressioni per la massa invariante si ottiene $E_e = m_\mu = \SI{0.106}{GeV}$ alla soglia.
Notare la grandissima differenza di energia necessaria per il fascio di elettroni nel primo e nel secondo caso.

\end{Answer}

\begin{Exercise}[title={Energia di soglia e decadimento in due corpi}]
  Si consideri la collisione frontale tra un fascio di protoni ed uno
  di elettroni, di pari impulso p nel sistema di riferimento del
  laboratorio, che produce la reazione:
  \beq
  e^- + p \to \Lambda + \nu_e
  \eeq
  \Question Determinare l’energia dell’elettrone quando la reazione e`
  prodotta a soglia con $\Lambda \to p \pi^-$.
  \Question Determinare l’impulso del protone e del pione, prodotti dal decadimento della
  $\Lambda$, nel sistema di riferimento in cui la $\Lambda$ \`e in quiete
  [$m_e = \SI{0.511}{MeV/c^2}$, $m_p = \SI{938.3}{MeV/c^2}$, $m_\Lambda = \SI{1115.7}{MeV/c^22}$, $m_\pi = {139.6}{MeV/c^2}$]
\end{Exercise}
\begin{Answer}
  Il sistema del laboratorio coincide col sistema del centro di massa. L’impulso di soglia si ricava quindi imponendo:
  \beq
  \sqrt{s}=\sqrt{p^2+m_e^2}+\sqrt{p^2+m_p^2}=m_\Lambda
  \eeq
  Elevando al quadrato si ha:
  \beq
  p^2+m_e^2+p^2+m_p^2+2\sqrt{(p^2+m_e^2)(p^2+m_p^2)}=m_\Lambda^2 \\
  \eeq
  Conviene portare al secondo membro la radice:
  \beq
  2p^2+m_e^2+m_p^2 - m_\Lambda^2 = 2\sqrt{(p^2+m_e^2)(p^2+m_p^2)}
  \eeq
  e poi elevando al quadrato:
  \beq
  4p^4+m_e^4+m_p^4+m_\Lambda^4+4p^2m_e^2+4p^2m_p^2-4p^2m_\Lambda^2 - 2m_e^2m_p^2-2m_e^2m_\Lambda^2-2m_p^2m_\Lambda^2 =
  4p^4+4p^2m_p^2+4p^2m_e^2+4m_e^2m_p^2
  \eeq
  si ottiene:
  \beq
  p^2=\frac{m_e^4+m_p^4+m_\Lambda^4-2me^2m_p^2-2m_e^2m\Lambda^2-2m_p^2m_\Lambda^2}{4m_\Lambda^2}
  \eeq
  e quindi il valore di $p=\SI{163}{MeV/c}$, da cui l'energia $E_e=pc=\SI{163}{MeV}$.

  Nel caso del successivo decadimento in due corpi $\Lambda\to p\pi^-$, si ottiene l'energia del protone, nel sistema di
  riferimento del centro di massa, usando la formula:
  \beq
  E_p^* = \frac{m_\Lambda^2+m_p^2-m_\pi^2}{2m_\Lambda} = \SI{944}{MeV}
  \eeq
  L'impulso del protone \`e quindi dato da:
  \beq
  p_p^* = \sqrt{{E_p^*}^2-m_p^2}=\SI{100}{MeV/c}
  \eeq
  Siccome nel centro di massa, per definizione, la somma vettoriale
  degli impulsi \`e nulla, l'impulso del pione carico bilancia quello
  del protone, emesso in direzione opposta, e quindi $p_\pi^*=p_p^* =\SI{100}{MeV/c}$.
\end{Answer}

\begin{Exercise}[title={Diffusione elastica di un fotone su un bersaglio}]
Chiamiamo elastico un urto (``scattering'') in cui le particelle dello
stato iniziale e dello stato finale sono le stesse. Si consideri un
urto elastico fra una particella di massa nulla e una particella di
massa $m$ (\emph{bersaglio}) che si trova a riposo nel sistema di
riferimento del laboratorio: qual è la massima energia trasferita
dalla particella incidente al bersaglio? \emph{Suggerimento: si lavori
nel sistema di riferimento del laboratorio, e si espliciti il prodotto
scalare fra gli impulsi spaziali della particella di massa nulla prima
e dopo l'urto in funzione dell'angolo, sempre nel sistema di
riferimento del laboratorio, fra la direzione iniziale e finale della
particella incidente.}

Se la particella incidente è un fotone e il bersaglio è un elettrone atomico a riposo, di quanto varia la lunghezza d'onda del fotone fra prima e dopo l'urto?
\end{Exercise}

\begin{Answer}
Per scattering elastico intendiamo un processo in cui le particelle dello stato iniziale sono le stesse di quelle dello stato finale.

Denotiamo con $\quadriv{k}$ e $\quadriv{P}$ i quadrimpulsi della particella incidente e del bersaglio prima dell'urto, e indichiamo con l'apice le stesse quantità dopo l'urto: il problema ci dice che
\begin{align*}
    \quadriv{k} = (E, \vett{k}),\\
    \quadriv{k'} = (E', \vett{k'}),\\
    \quadriv{P} = (mc, \vett{0}).\\
\end{align*}
Partiamo dalla conservazione del quadrimpulso durante l'urto, isoliamo la quantità che non misuriamo direttamente -- cioè il quadrimpulso del bersaglio dopo l'urto, $\quadriv{P'}$ -- ed eleviamo al quadrato:
\begin{align*}
    \quadriv{k} + \quadriv{P} = \quadriv{k'} + \quadriv{P'},\\
    \quadriv{P'} = \quadriv{k} + \quadriv{P} - \quadriv{k'},\\
    m^2c^2 = 0 + m^2c^2 + 0 + 2Em-2(EE'-\vett{k}\cdot\vett{k'})-2mE',
\end{align*}
e se indichiamo con $\theta'$ l'angolo -- nel riferimento del laboratorio -- fra la direzione iniziale e finale della particella incidente, e usiamo il fatto che $|\vett{k}|c = E$ e $|\vett{k'}|c = E'$,
\begin{align*}
    0= 2mc^2(E-E')-2(EE'-EE'\cos\theta'),\\
    mc^2(E'-E) = -EE'(1-\cos\theta'),\\
    E'(mc^2 + E(1-\cos\theta')) = mc^2E,\\
    E' = \frac{E}{1 + \frac{E}{mc^2}(1-\cos\theta')}.
\end{align*}
Il bersaglio rinculerà di una energia $E-E'$, massima per $\theta=\pi$. Il valore massimo di quest'energia di rinculo,
\begin{equation*}
    E-\frac{E}{1+2\frac{E}{mc^2}} = E\frac{2E/mc^2}{1+2E/mc^2},
\end{equation*}
prende il nome -- nel caso dello scattering Compton, in cui la particella incidente è un fotone e il bersaglio è un elettrone atomico -- di \emph{picco Compton}.

Cosa cambia fra un fotone di energia $E$ ed uno di energia $E'$? Dalla meccanica quantistica,
\begin{equation*}
    E = h\nu = \frac{hc}{\lambda},
\end{equation*}
cioè cambia la lunghezza d'onda del fotone:
\begin{align*}
    E' = \frac{hc}{\lambda'} = \frac{\frac{hc}{\lambda}}{1+\frac{\frac{hc}{\lambda}}{mc^2}(1-\cos\theta')},\\
    \frac{1}{\lambda'} = \frac{\frac{1}{\lambda}}{1+\frac{\frac{hc}{\lambda}}{mc^2}(1-\cos\theta')},\\
    \lambda' = {\lambda}\left(1+\frac{hc}{\lambda mc^2}(1-\cos\theta')\right),\\
    \lambda' = \lambda + \frac{h}{m c}(1-\cos\theta')\equiv \lambda + \lambda_c(1-\cos\theta').
\end{align*}
dove abbiamo definito la \emph{lunghezza d'onda Compton} dell'elettrone, $\lambda_c$, che rappresenta la scala di lunghezza sotto la quale gli effetti della meccanica quantistica relativistica divengono importanti.
\end{Answer}

\begin{Exercise}[title={Scattering Rutherford}]
Un fascio di particelle $\alpha$ di \SI{100}{MeV} di energia e \SI{0.32}{nA} di corrente\footnote{Per una spiegazione breve su come (e perché) si misura la corrente di un fascio di particelle, vedi \url{https://www.lhc-closer.es/taking_a_closer_look_at_lhc/0.beam_current}. Una trattazione più completa è data ad esempio da \url{https://cds.cern.ch/record/1213275/files/p141.pdf}.} collide contro un bersaglio fisso di alluminio, spesso \SI{1}{cm}. Una sperimentatrice prende un rivelatore di $\SI{1}{cm}\times\SI{1}{cm}$ di superficie, e lo posiziona ad un angolo di \ang{30} rispetto al fascio di particelle, a \SI{1}{m} di distanza dal bersaglio. Quante particelle $\alpha$ incideranno sul rivelatore ogni secondo?
\end{Exercise}
\begin{Answer}
L'alluminio ha una densità di \SI{2.7}{g/cm^3}, numero atomico $13$ e massa atomica \SI{27}{u}.

Poiché le particelle $\alpha$ sono nuclei di elio, hanno carica $2e$ e la corrente di \SI{0.32}{nA} corrisponde a un miliardo di particelle incidenti al secondo,
\[
\dv{N_i}{t}=\frac{\SI{0.32}{nC/s}}{2\times\SI{1.6e-19}{C}} = \SI{1e9}{s^{-1}}.
\]

Il rivelatore vede un angolo solido di
\[
\Delta\Omega\equiv \frac{\text{superficie}}{\text{raggio}}^2 = \frac{\SI{1}{cm^2}}{(\SI{1}{m})^2} = \SI{1e-4}{sr}
\]

Si tratta di uno scattering alla Rutherford, per cui la sezione d'urto per unità di angolo solido rilevata ad un certo angolo $\theta$ vale
\[
\dv{\sigma}{\Omega} = \left(\frac{z_{\alpha}z_{Al}e^2}{4\pi\epsilon_04E}\frac{1}{\sin^2(\theta/2)}\right)^2,
\]
pari a
\begin{equation}\begin{split}
\dv{\sigma}{\Omega} \approx \left(\frac{2\times13\times4\times e\times\SI{1.6e-19}{C}}{4\pi\times\SI{8.9e-12}{F/m}\times4\times\SI{100e6}{eV}}\frac{1}{\sin^2(\pi/\ang{180}\times\ang{30}/2)}\right)^2\\
\approx \SI{2e-30}{m^2/sr}=\SI{20}{mb/sr},
\end{split}\end{equation}
e il numero di particelle visto dal rivelatore vale, se indichiamo con $n_{Al}=\rho_{Al}\frac{N_A}{A_{Al}}$ la densità numero di atomi di alluminio, e con $d$ lo spessore del rivelatore,
\begin{equation*}
\begin{split}
&\dv{N_\text{rivelate}}{t}=\Delta\Omega\dv{\sigma}{\Omega}n_{Al}d \dv{N_\text{i}}{t}\\ 
&\approx\SI{1e-4}{sr}\times \SI{2e-30}{m^2/sr} \times \SI{1e4}{cm^2/m^2}  \times \SI{2.7}{g/cm^3} \frac{\SI{6e23}{mol^{-1}}}{\SI{27}{g/mol}}\\&= \SI{120}{Hz}.
\end{split}
\end{equation*}
\end{Answer}

%\begin{Exercise}[title={Perdita di energia per ionizzazione}]
In un centro di radioterapia, degli elettroni sono accelerati da un acceleratore lineare fino a un'energia di \SI{25}{MeV}. 

\Question Calcolare l'energia che depositano in \SI{1}{mm} di tessuto umano, assumendo per esso caratteristiche pari a quelle dell'acqua.

\Question Quanto piombo è necessario per ridurre l'energia degli elettroni fino ad un valore pari all'energia critica del piombo? Si trascurino le perdite di energia per ionizzazione.

\Question Trascurando le perdite di energia per irraggiamento al di sotto dell'energia critica, qual è lo spessore di piombo aggiuntivo necessario a \emph{fermare} gli elettroni, assumendo conservativamente che la loro perdita di energia per ionizzazione nel piombo sia costante e pari a circa \SI{11}{MeV/cm}? Si assuma che per gli elettroni valga la normale formula di Bethe-Bloch,
\[
-\dv{E}{x} = C\rho\left(\frac{z}{\beta}\right)^2\frac{Z}{A}\left[\log\frac{2m_ec^2(\beta\gamma)^2}{I}-\beta^2-\delta/2\right],
\]
e che:
\begin{itemize}
    \item acqua: $\rho=\SI{1}{g/cm^3}$, $\langle I\rangle=\SI{80}{eV}$, $E_c=\SI{80}{MeV}$, $X_0=\SI{36.1}{cm}$, $Z/A=0.55$; per elettroni da \SI{25}{MeV} in acqua, $\delta/2=4.5$;
    \item piombo: $\rho=\SI{11.35}{g/cm^3}$, $\langle I \rangle=\SI{823}{eV}$, $E_c=\SI{7.4}{MeV}$, $X_0=\SI{0.56}{cm}$, $Z/A=0.40$, $\delta/2=0.3$.
\end{itemize}
\end{Exercise}
\begin{Answer}
Gli elettroni hanno
\begin{align*}
\beta&=\frac{p}{E}\approx0.99989\\
\beta\gamma&=\frac{p}{m}\approx48.9.
\end{align*}
\begin{itemize}
    \item In \SI{1}{mm} di tessuto umano, perderanno per ionizzazione -- assumendo che anche per gli elettroni valga la formula di Bethe -- un'energia
    \[
    \Delta E = \dv{E}{x}\Delta x  \approx \SI{200}{keV},
    \]
    mentre perderanno per irraggiamento un'energia molto minore,
    \[
    \Delta E = E_0\left(1-\exp(-\frac{\Delta x}{X_0})\right)\approx \SI{70}{keV},
    \]
    consistentemente col valore dell'energia critica $E_c$.

    \item Perché la sua energia scenda ad $E_c^\text{Pb}$, l'elettrone deve perdere
    \[
    \Delta E = E-E_c^\text{Pb} = \SI{17.6}{MeV},
    \]
    che per solo irraggiamento vengono persi dopo una distanza $\Delta x$ tale che
    \[
    \frac{E-\Delta E}{E} = \exp(-\frac{\Delta x}{X_0^\text{Pb}}),
    \]
    da cui segue $\Delta x \approx\SI{6.8}{mm}$.
\end{itemize}
\end{Answer}

\begin{Exercise}[title={Perdita di energia per ionizzazione}]
  Nell'atmosfera si possono formare e propagare degli sciami estesi di raggi cosmici costituiti essenzialmente di fotoni, elettroni e muoni, questi ultimi chiamati {\textit componente dura} dello sciame.

  \Question Calcolare l'energia persa dai muoni in uno sciame, se essi hanno un'energia pari a $E_\mu=\SI{1000}{GeV}$ nell'attraversare uno spessore di roccia di \SI{1}{cm}. Si assuma per la roccia una densit\`a $\rho=\SI{3.0}{g/cm^3}$; $Z/A=\frac{1}{2}$ e il potenziale medio di ionizzazione $\langle I \rangle=\SI{200}{eV}$.

  \Question Il fascio di muoni cosmici di energia $E=\SI{1000}{GeV}$ incide verticalmente sulla superficie del terreno e si assume per semplicit\`a che nella roccia si abbia:  $\frac{1}{\rho}\frac{dE}{dx}=$costante$=\SI{2}{MeV/g cm^2}$. Calcolare lo spessore di roccia che riduce in quiete tali muoni.  
\end{Exercise}
\begin{Answer}
  Per i muoni usiamo la formula di Bethe-Bloch approssimata (senza effetto densit\`a e correzione di shell):
  \[
-\dv{E}{x} = C\rho\left(\frac{z}{\beta}\right)^2\frac{Z}{A}\left[\log\frac{2m_ec^2(\beta\gamma)^2}{I}-\beta^2\right],
\]
dove:
\begin{itemize}
\item la costante $C=4\pi r_e^2m_ec^2N_A = \SI{0.307}{MeV/g cm^2}$;
\item la densit\`a della roccia \`e data dal problema: $\rho=\SI{3.0}{g/cm^3}$;
\item il rapporto $Z/A=\frac{1}{2}$ \`e l'approssimazione tipica che facciamo (numero di protoni $\approx$ numero di neutroni in un nucleo);
\item il potenziale medio di ionizzazione $\langle I \rangle=\SI{200}{eV}$ \`e dato.
\end{itemize}
Quindi si possono calcolare i fattori relativistici da mettere nella formula di Bethe-Bloch:
\[
\beta_\mu^2 = 0.9999 \rightarrow  \gamma_\mu = 9464 \rightarrow  \gamma_\mu^2 \approx 90 \cdot 10^6
\]
e quindi:
\[
\frac{dE}{dx} = \SI{11.90}{MeV/cm}
\]
Siccome stiamo guardando uno strato piccolo di roccia $\Delta x$ in un regime circa di \textit{minimum ionizing particle}, nell'attraversare uno strato di profondit\`a $d=\SI{1}{cm}$ otteniamo:
\[
\Delta E = \frac{dE}{dx} \cdot \Delta x = \SI{11.9}{MeV}.
\]

Fermare i muoni significa che essi perdono tutta la loro energia cinetica $K = E_\mu - m_\mu$ per ionizzazione. Se la perdita di energia \`e circa costante (assunzione del problema), allora:
\[
\frac{1}{\rho}\frac{\Delta E}{\Delta x} = \SI{2}{MeV/g cm^2}
\]
e quindi ($\Delta E = K$):
\[
\Delta x = \frac{K}{\rho \left[\frac{g}{cm^3}\right]} \cdot \frac{1}{\SI{2}{MeV \frac{cm^2}{g}}}
\]
e quindi:
\[
\Delta x  = \frac{10^6-105.66}{3 \cdot 2} = \SI{1.67}{km}.
\]
\end{Answer}



\begin{Exercise}[title={Spettrometro, ionizzazione, multiplo scattering, assorbimento}]
Un fascio contenente muoni e pioni carichi di impulso pari a \SI{1}{GeV/c} attraversa un campo magneteico di \SI{0.57}{T}. Successivamente incide su due scintillatori di NaI(Tl) di spessore $d = \SI{5}{cm}$, posti a distanza $D = \SI{5}{m}$ uno dall’altro.

\Question Calcolare il raggio di curvatura della traiettoria nel campo magnetico;

\Question Calcolare l’energia depositata nel primo scintillatore rispettivamente da pioni e muoni (si trascuri il termine $\delta(\gamma)$ nella formula di Bethe-Bloch) ed il tempo di volo tra i due scintillatori;

\Question Calcolare la deviazione media rispetto alla traiettoria centrale con cui i muoni arrivano sul secondo scintillatore, a causa dello scattering multiplo nel primo scintillatore;

\Question Per attenuare il fascio di pioni, si interpone un assorbitore in piombo tra i due scintillatori. Assumendo per i pioni in questione una lunghezza di interazione nel piombo di \SI{20}{cm}, si determini lo spessore necessario affinch\'e il 50\% dei pioni interagisca prima di arrivare sul secondo scintillatore.

Si usino:
\begin{itemize}
\item $m_\pi=\SI{139.6}{MeV/c^2}$, $m_\mu = \SI{105.7}{MeV/c^2}$
\item Per il mezzo, NaI(Tl), si usino  $\rho = \SI{3.67}{g/cm^3}$, $I = \SI{452}{eV}$, $X_0 = \SI{2.59}{cm}$, $Z/A = 0.45$.
\end{itemize}
\end{Exercise}
\begin{Answer}
  \begin {enumerate}

  \item In unint\`a in cui il campo magnetico \`e misurato in [T] e l'impulso delle particelle in GeV, il raggio di curvatura (misurato in [m]) di una particella carica che viaggia in un piano ortogonale all'asse del campo magnetico \`e:
    \beq
    R = \frac{p [GeV]}{0.3 B [T]} = \SI{5.85}{m}
    \eeq

  \item I pioni di impulso \SI{1}{GeV} hanno $\beta_\pi = 0.990$ e $\beta_\pi\gamma_\pi = 7.16$, mentre i muoni $\beta_{\mu} = 0.994$ e $\beta_\mu\gamma_\mu = 9.47$. La loro perdita di energia nel primo scintillatore calcolata con la formula di Bethe-Bloch vale:
    \begin{align*}
      -\dv{E}{x}(\pi) &= \SI{5.52}{MeV/cm} \\
      -\dv{E}{x}(\mu) &= \SI{5.76}{MeV/cm}
    \end{align*}
    Quindi attraversando lo spessore di $d = \SI{5}{cm}$ di NaI(Tl) essi perdono un'energia pari a $\Delta E_{\pi} = \SI{27.6}{MeV}$ per i pioni e $\Delta E_{\mu} = \SI{28.8}{MeV}$ per i muoni.
    Il loro impulso dopo il primo scintillatore sar\`a:
    \begin{align*}
      p_\pi = \sqrt{(E_i-\Delta E)^2-m_\pi^2} = \sqrt{\left(\sqrt{p_i^2+m_\pi^2} - \Delta E\right)^2 - m_\pi^2} = \SI{0.972}{GeV} \\
      p_\mu = \sqrt{(E_i-\Delta E)^2-m_\mu^2} = \sqrt{\left(\sqrt{p_i^2+m_\mu^2} - \Delta E\right)^2 - m_\mu^2} = \SI{0.971}{GeV}
    \end{align*}
    Dall'impulso possiamo ottenere la velocit\`a delle particelle, $\beta_\pi=0.990$ e $\beta_\mu=0.994$. Il tempo di volo tra gli scintillatori sar\`a quindi pari a:
    \begin{align*}
      \Delta T_\pi = \frac{D}{\beta_\pi c} = \SI{16.8}{ns} \\
      \Delta T_\mu = \frac{D}{\beta_\mu c} = \SI{16.8}{ns}
    \end{align*}
    notare che con questa distanza tra gli scintillatori e cinematica delle particelle \`e impossibile determinare l'ipotesi di massa della particella carica dal solo tempo di volo.

  \item Lo scattering coulombiano multiplo sar\`a mediamente di un angolo pari a:
    \beq
    \langle \theta_{MS} \rangle = \SI{21}{MeV} \frac{z}{\beta p} \sqrt{\frac{x}{X_0}} = \SI{29.5}{mrad}
    \eeq
    sia per i pioni che per i muoni, portando a una deviazione media all’altezza del secondo scintillatore pari a:
    \beq
    \langle \delta x \rangle = D \tan(\theta_{MS}) \approx D \theta_{MS} = \SI{14.7}{cm}.
    \eeq

  \item Con l’assorbitore il fascio di pioni si riduce di un fattore:
    \beq
    \frac{\Phi}{\Phi_0} = e^{-x/\lambda_{int}} = 0.5
    \eeq
    dal quale, facendo il logaritmo di entrambi i lati dell'equazione, $x = \SI{13.9}{cm}$.
  \end{enumerate}

\end{Answer}

%\input{prova-itinere-01-28-04-2021.tex}
%\begin{Exercise}
  Il rivelatore SuperKamiokande \`e un grosso cilindro verticale riempito di acqua (indice di rifrazione $n=1.33$,
  densit\`a $\rho=\SI{1}{g/cm^3}$). I neutrini atmosferici di tipo muonico ($\nu_\mu$) sono studiati rivelando la luce
  Cherenkov emessa dai muoni prodotti nell'interazione con i nuclei dell'acqua. Assumendo che si vogliono misurare soltanto
  i muoni di impulso fino a $p_\textrm{max}=\SI{1}{GeV/c}$ si determini:
  \Question qual \`e il percorso massimo dei muoni nell'acqua, considerando un percorso rettilineo e trascurando
  l'effetto del multiplo scattering
  \Question per quale parte di questo percorso il muone emette luce Cherenkov
  \Question qual \`e il raggio del cerchio illuminato sulla base del cilindro da un muone di impulso $p_\mu=\SI{1}{GeV/c}$ prodotto
  lungo l'asse del cilindro, ad un'altezza $h=\SI{50}{cm}$ dalla base stessa
\end{Exercise}
\begin{Answer}
  \begin{enumerate}
  \item La perdita di energia per i muoni nell'acqua \`e principalmente dovuta alla ionizzazione, mentre il
    contributo della perdita di energia per emissione di luce Cerenkov \`e circa mille volte inferiore.
    I muoni hanno massa $m_\mu=\SI{106}{MeV/c^2}$, e quindi il loro fattore di Lorentz \`e:
    \[
    \beta\gamma = \frac{p}{m_\mu c}=\frac{\SI{1}{GeV/c}}{\SI{0.106}{GeV/c}} = 9.4
    \]
    A questo valore la perdita di energia \`e ancora ben approssimata
    dal valore al minimo della ionizzazione (\`e solo l'inizio della
    salita relativistica, che \`e logaritmica). Quindi si pu\`o
    assumere una perdita di energia circa costante, e pari a
    $\dv{E}{x}\vert_\textrm{min}\approx\SI{2}{MeV/g \cdot cm^{-2}}$.
    L'energia cinetica massima dei muoni che si vogliono considerare \`e:
    \[
    K_\textrm{max} = \sqrt{p_\textrm{max}^2+m_\mu^2} - m_\mu \approx \SI{900}{MeV}
    \]
    e quindi il percorso massimo \`e dato da:
    \[
    d_\textrm{max} = \frac{K_\textrm{max}}{\rho\cdot\dv{E}{x}\vert_\textrm{min}} \approx \SI{4.5}{m}
    \]

  \item Il muone emette luce Cherenvov finch\'e il suo impulso \`e superiore alla soglia:
    \[
    \beta > \beta_\textrm{min} = \frac{1}{n} \approx 0.75
    \]
    e quindi:
    \beqn
    \frac{p}{E} &>& \beta_\textrm{min} \Rightarrow \frac{p}{p^2+m_\mu^2} > \beta_\textrm{min}
    \Rightarrow p^2 >  \beta_\textrm{min} (p^2+m_\mu^2) \\
    &\Rightarrow& p>\frac{\beta_\textrm{min} m_\mu}{\sqrt{1-\beta_\textrm{min}}} \approx \SI{120}{MeV/c} \equiv p_\textrm{min}
    \eeqn
    e quindi l'energia cinetica minima \`e:
    \[
    K_\textrm{min} = \sqrt{p_\textrm{min}^2+m_\mu^2} - m_\mu^2 \approx \SI{54}{MeV}
    \]
    Mentre il muone viaggia nell'acqua, perde energia per
    ionizzazione, come fatto nel punto precedente, fino a raggiungere
    $K_\textrm{min}$. Sempre approssimando la perdita di energia come
    una costante (che \`e ancora un'approssimazione abbastanza buona per $p_\textrm{min}\approx \SI{100}{MeV}$), si ha
    che il percorso durante il quale il muone emette luce Cherenkov \`e:
    \[
    L = \frac{K_0 -
      K_\textrm{min}}{\rho\cdot\dv{E}{x}\vert_\textrm{min}} =
    \frac{\SI{900}{MeV}-\SI{54}{MeV}}{\SI{2}{MeV/cm}} \approx
    \SI{4.2}{m}
    \]
    Quindi la frazione del percorso massimo dei muoni in cui questo avviene \`e:
    \[
    f = \frac{L}{d_\textrm{max}} = \frac{\SI{4.2}{m}}{\SI{4.5}{m}} = 93\%
    \]

  \item Un mone di impulso \SI{1}{GeV/c} emette luce Cherenkov su un cono di apertura:
    \[
    \cos\theta_C = \frac{1}{\beta n}
    \]
    dove $\beta = p/E = p/\sqrt{p^2+m_\mu^2}=0.9944$ e quindi 
    \[
    \cos\theta_C = \frac{1}{0.9944\cdot 1.33} \approx 0.756
    \]
    durante il suo percorso verso la base, il vertice di emissione
    della luce si avvicina, e quindi i successivi coni di luce sono
    contenuti nel primo. Inoltre, perdendo energia, il $\beta$
    diminuice, e quindi $\cos\theta_C$ aumenta, e quindi anche
    $\theta_C$ diminuisce, cntribuendo a rendere i coni successivi
    contenuti nel primo.  Quindi la zona illuminata \`e interna al
    cerchi dell'emissione avvenuta alla distanza $h=\SI{50}{cm}$ dalla
    base:
    \[
    R = h \tan\theta_C = h \frac{\sqrt{1-\cos^2\theta_C}}{\cos\theta_C} \approx 0.865 \cdot h \approx \SI{43}{cm}
    \]
  \end{enumerate}
\end{Answer}

\begin{Exercise}
  Un fascio di pioni carichi interagisce con un bersaglio di protoni
  fermi nel sistema di riferimento del laboratorio. Si vuole studiare
  la reazione $\pi^- + p \rightarrow \Lambda^0 + L^0$. Calcolare
  l'energia minima del pione affinch\'e la reazione possa avvenire.
  Le masse delle particelle della reazione sono:
  $m_\pi=\SI{140}{MeV/c^2}$, $m_p=\SI{938}{MeV/c^2}$,
  $m_\Lambda=\SI{1116}{MeV/c^2}$, $m_K=\SI{498}{MeV/c^2}$
  
  \Question Se l'energia del fascio \`e $E_\pi=\SI{2.0}{GeV}$,
  calcolare, se esiste, l'angolo massimo nel laboratorio con cui viene
  emessa la particella $\Lambda^0$. 
\end{Exercise}
\begin{Answer}
  L'energia di soglia del pione \`e la minima necessaria per produrre
  le particelle finali a riposo nel sistema di riferimento del centro
  di massa ($E_\textrm{min}$). Uguagliando la massa invariante dello
  stato iniziale e finale nel centro di massa: \beqn
  \sqrt{s}_\textrm{in} = \sqrt{E_\textrm{min} + m_p} \\
  \sqrt{s}_\textrm{fin} = \sqrt{m_K + m_\Lambda} \eeqn elevando al
  quadrato ed uguagliando le masse invarianti si ottiene:
  \beqn
  E^2_\textrm{min} + m_p^2 + 2E_\textrm{min}m_p =  (m_K + m_\Lambda)^2 \\
  \Rightarrow E_\textrm{min} = \frac{(m_K + m_\Lambda)^2 - m_\pi^2 - m_p^2}{2m_p} \approx \SI{0.91}{GeV}
  \eeqn

  L'angolo massimo di emissione nel laboratorio esiste solo nel caso in cui la velocit\`a della particella
  nel centro di massa, $\beta^*$, \`e minore della velocit\`a del centro di massa nel laboratorio, $\beta^{cm}$:
  \[
  \beta^* < \beta^{cm} = \frac{p_{cm}}{E_{cm}} = \frac{\sqrt{E_\pi^2-m_\pi^2}}{E_\pi+m_p} \approx 0.665
  \]
  Per verificare se la condizione \`e possibile bisogna calcolare l'impulso e l'energia della $\Lambda^0$ nel centro di massa.
  L'energia totale del centro di massa \`e:  
  \[
  E^*_\textrm{tot} = \sqrt{m_p^2+m_\pi^2+2E_\pi m_p} \approx \SI{2.16}{GeV}
  \]
  Usando la conservazione dell'energia nel centro di massa, e chiamando $p^* \equiv \vert \vec p^*_\Lambda\vert = \vert \vec p^*_K\vert$,
  si pu\`o calcolare l'impulso della $\Lambda^0$ nel centro di massa:
  \[
  E^*_\textrm{tot} = E^*_\Lambda + E^*_K = \sqrt{{p^*}^2+m_\Lambda^2} + \sqrt{{p^*}^2+m_K^2}
  \]
  elevando due volte al quadrato e risolvendo l'equazione in $p^*$ si ottiene
  \[
  p^* = \frac{\sqrt{[{E^*}^2-(m_\Lambda + m_K)^2][{E^*}^2-(m_\Lambda - m_K)^2]}}{2E^*} \approx \SI{0.69}{GeV/c}
  \]
  da cui si ricava:
  \[
  \beta^* = \frac{p^*}{E^*} = \frac{p^*}{{p^*}^2+m_\Lambda^2} \approx 0.52.
  \]
  Quindi la condizione $\beta^*<\beta_{cm}$ \`e soddisfatta, per cui esiste un angolo massimo di emissione della particella $\Lambda^0$ nel laboratorio.
  Questo angolo \`e:
  \[
  \theta_{max} = \atan \left\{ \left[ \gamma_{cm}\sqrt{\left(\frac{\beta_{cm}}{\beta^*}\right)^2-1}\right]^{-1}\right\} \approx 42^\circ
  \]
\end{Answer}

%\begin{Exercise}[title={Densit\`a nucleare e unit\`a di misura}]
  Si stimi la densit\`a nucleare in $\SI{}{g cm^3}$, approssimando $m_p = m_n = \SI{938.3}{MeV/c^2}$.
\end{Exercise}
\begin{Answer}
Un elettronVolt (eV) \`e l'energia cinetica acquistata da una
particella di carica elementare $e =\SI{1.602e-19}{C}$ che passa
attraverso la differenza di potenziale di un Volt, per cui:
\[
\SI{1}{eV} = \SI{1.602e-19}{J}.
\]
Il fattore di conversione tra massa (che in relativit\`a \`e equivalente a un'energia) \`e dato da:
\[
E(\SI{1}{kg} = \SI{1}{kg c^2} = \SI{9e16}{kg(m/s)^2} = \SI{9e16}{J}
\]
e poich\'e $\SI{1}{J}=1/\SI{1.6e-19}{eV}$ si ha:
\[
\SI{1}{kg c^2} = \frac{9e16}{1.6e-19}\SI{}{eV} = \SI{5.6e35}{eV} 
\]
da cui:
\[
\SI{1}{kg} = \SI{5.6e35}{eV/c^2} 
\]
Quindi la massa del protone, espressa in kg, vale:
\[
m_p = \SI{9.383e8}{eV/c^2} = \frac{\SI{9.383e8}{}}{\SI{5.6e35}{}}\SI{}{kg} = \SI{1.67e-27}{kg}
\]

Il testo dice di considerare $m_n \approx m_p = \SI{1.67e-24}{kg}$. Quindi usando il raggio del nucleo:
\[
R = R_0 \cdot A^{1/3}
\]
con $R_0=\SI{1.2}{fm}$, che \`e valido per grandi valori di $A$, si ottiene:
\beqn
\rho &=& \frac{M}{V} = \frac{A\cdot m_p}{4/3 \pi R_0^3 A} = \frac{3 m_p}{4\pi R_0^3} \\
&=& \frac{3 \times \SI{1.67e-27}{g}}{12.56 \times (\SI{1.2e-13}{cm})^3} \approx \SI{2.3e14}{g/cm^3}
\eeqn
\end{Answer}

\begin{Exercise}[title={Termine coulombiano della formula di Weiszacker}]
Considerando che l'energia elettrostatica di una carica $Q$
uniformemente distribuita su una sfera di raggio $R$ \`e uguale a
$\frac{3}{5}\cdot\frac{Q^2}{4\pi\epsilon_0 R}$, stimare il termine
coulombiano della formula semi-empirica delle masse dei nuclei.
\end{Exercise}
\begin{Answer}
La carica del nucleo \`e $Q=Ze$, mentre il raggio del nucleo si pu\`o stimare con:
\[
R = R_0 \cdot A^{1/3}
\]
dove $R_0=\SI{1.2}{fm}$, che \`e valido per grandi valori di $A$.
Il termine coulombiano della formula di Weiszacker \`e quello proporzionale a $1/R$, quindi:
\[
\frac{3Z^2e^2}{20\pi \epsilon_0 R_0A^{1/3}} = a_C \cdot \frac{Z^2}{A^{1/3}}
\]
da cui si ricava il coefficiente $a_C$ del termine coulombiano, fattorizzandolo in modo opportuno da
evidenziare delle costanti di cui sappiamo il valore:
\[
a_C = \frac{3}{5}\times\frac{e^2}{4\pi\epsilon_0}\times\frac{1}{R_0} =
0.6\times k e^2 \times \frac{1}{\SI{1.2}{fm}} 
\]
usando il valore della costante di Coulomb:
\[
k e^2 = \frac{\SI{197}{MeV \cdot fm}}{137}=\SI{1.44}{MeV \cdot fm}
\]
si ottiene:
\[
a_C \approx \SI{0.7}{MeV}
\]
che \`e una buona approssimazione per tale costante.
\end{Answer}

\begin{Exercise}[title={Termine coulombiano della formula di Weiszacker}]
  Gli stati stabili di \ce{^{13}_{6}C} e \ce{^{13}_7N} appartengono
  allo stesso doppietto di isospin, e la loro differenza di massa \`e
  dovuta principalmente dalla diversa energia coulombiana, e in modo
  sottodominante dalla differenza di massa tra neutrone e protone.

  \Question Considerando entrambe le cause di differenza di massa, si
  stimi la differenza di massa dei due nuclei.
\end{Exercise}
\begin{Answer}
  Come nell'esercizio precedente, approssimando la carica elettrostatica come uniformemente
  distribuita in delle sfere di raggio $R=R_0 A^{1/3}$, con $R_0 = \SI{1.2}{fm}$, la loro
  energia elettrostatica \`e, per una carica $Q$:
  \[
  E_C = \frac{3}{5}\times\frac{Q^2}{4\pi\epsilon_0 R}
  \]
  Poich\'e il \ce{^{13}_7N} ha un protone in pi\`u rispetto al \ce{^{13}_{6}C}, esso avr\`a un'energia coulombiana maggiore.
  Per\`o bisogna considerare che il neutrone ha minor massa rispetto al protone:
  \[
  M_n - M_p - m_e = \SI{0.782}{MeV/c^2}
  \]
  Quindi la differenza di massa tra i due nuclei \`e data da:
  \beqn
      [M(\ce{^{13}_7N}) - M(\ce{^{13}_{6}C})] &=& \frac{3}{5R\times 4\pi\epsilon_0}(Q^2_N-Q^2_C) - [M_n - M_p - m_e] \\
      &=& \frac{3}{5R}\left(\frac{e^2}{4\pi\epsilon_0}\right)(7^2-6^2) - \SI{0.782}{MeV} \\
      &=& 0.6 \times \SI{1.44}{MeV\cdot fm} \times \frac{49-36}{\SI{1.2}{fm}\times 13^{1/3}} - \SI{0.782}{MeV} \\
      &\approx& \SI{2.62}{MeV}
  \eeqn
\end{Answer}

\begin{Exercise}[title={Applicazione della legge di Geiger-Nuttal al decadimento $\alpha$}]
  La relazione di Geiger-Nuttal lega in modo semplice la costante di decadimento ($\lambda$) della
  radioattivit\`a naturale $\alpha$ e l'energia della particella $\alpha$ ($E_\alpha$) emessa.

  \Question Usando tale legge, si stimi l'ordine di grandezza per il
  tempo di dimezzamento del \ce{^{210}Po}, che decade emettendo
  $E_\alpha=\SI{5.3}{MeV}$, sapendo che il \ce{^{214}Po} ha un tempo
  di dimezzamento di $\SI{1.6e-4}{s}$, ed emette una particella
  $\alpha$ con $E_\alpha={7.7}{MeV}$.
\end{Exercise}
\begin{Answer}
  Per radionuclidi della stessa serie che decadono emettendo particelle $\alpha$, la legge di Geiger--Nuttal lega in
  modo lineare $\ln \lambda$ e $1/\sqrt{E_\alpha}$. Per il \ce{_{84}Po} si ha:
  \[
  \ln \lambda \approx a - \frac{Z-2}{E_\alpha}b
  \]
  Per i due isotopi del \ce{_{84}Po}, che hanno lo stesso $Z=84$, si ha:
  \beqn
  \ln \lambda(\ce{^{210}Po}) &\approx& a - \frac{Z-2}{E_\alpha(\ce{^{210}Po})}b \\
  \ln \lambda(\ce{^{214}Po}) &\approx& a - \frac{Z-2}{E_\alpha(\ce{^{214}Po})}b
  \eeqn
  poich\'e le costanti $a$ e $b$ sono debolmente dipendenti per gli
  isotopi, sottraendo dalla seconda la prima equazione si ottiene:
  \[
  \ln \lambda(\ce{^{214}Po}) - \ln \lambda(\ce{^{210}Po}) = b (Z-2)\left[\frac{1}{E_\alpha(\ce{^{210}Po})} - \frac{1}{E_\alpha(\ce{^{210}Po})}\right]
  \]
  ovvero:
  \[
  \ln \frac{\lambda(\ce{^{214}Po})}{\lambda(\ce{^{210}Po})} = b (Z-2)\left[\frac{1}{\sqrt{E_\alpha(\ce{^{210}Po})}} - \frac{1}{\sqrt{E_\alpha(\ce{^{210}Po})}}\right]
  \]
  Sappiamo che la costante $b$ \`e pari a:
  \[
  b \equiv \frac{e^2\sqrt{2m_\alpha}}{2\hbar\epsilon_0} = \frac{2\pi e^2\sqrt{2m_\alpha}}{4\pi\hbar\epsilon_0}
  =2\pi \frac{ke^2\sqrt{2m_\alpha c^2}}{\hbar c}= 6.28 \frac{\SI{1.44}{MeV\cdot fm} \cdot \sqrt{2\cdot\SI{3727}{MeV/c^2}}}{\SI{197}{MeV\cdot fm}} \approx \SI{4}{MeV^{1/2}}
  \]
  Per avere una stima del rapporto in ordini di grandezza tra i due tempi di dimezzamento consideriamo che:
  \[
  \ln \frac{\lambda(\ce{^{214}Po})}{\lambda(\ce{^{210}Po})} = \ln \frac{t_{1/2}(\ce{^{210}Po})}{t_{1/2}(\ce{^{214}Po})}
  \]
  e quindi
  \beqn
  \log_{10}\frac{t_{1/2}(\ce{^{210}Po})}{t_{1/2}(\ce{^{214}Po})} &=& \frac{1}{\ln(10)} \ln \frac{t_{1/2}(\ce{^{210}Po})}{t_{1/2}(\ce{^{214}Po})} \\
  &\approx&0.434 \times  \SI{4}{MeV^{1/2}} \times(84-2)\times
  \left[\frac{1}{\sqrt{\SI{5.3}{MeV}}} - \frac{1}{\sqrt{\SI{7.7}{MeV}}}\right] \approx 10.5
  \eeqn
  Le vite medie dei due nuclidi differiscono di ben 10 ordini di grandezza. Conoscendo quindi il tempo di dimezzamento
  del \ce{^{214}Po} si trova il valore numerico della vita media del \ce{^{210}Po}:

  \[
  t_{1/2}(\ce{^{210}Po}) \approx 10^{10.5} \cdot t_{1/2}(\ce{^{214}Po}) \approx 10^{10} \cdot 10^{1/2} \cdot \SI{1.6e6}{s} = \SI{58.6}{giorni}.
  \]

  Il valore sperimentale per il tempo di dimezzamento del
  \ce{^{210}Po} \`e circa 140 giorni, che \`e dello stesso
  ordine di grandezza del valore stimato con la legge di
  Geiger-Nuttal.
\end{Answer}

\begin{Exercise}[title={Decadimenti in sequenza}]
 Si consideri la seguente sequenza di decadimento:
\begin{itemize}
\item $N_1 \to N_2$, con costante di decadimento $\omega_1 = 10~s^{-1}$;
\item $N_2 \to N_3$, con costante di decadimento $\omega_2 = 50~s^{-1}$;
\item $N_3$ \`e stabile.
\end{itemize}

\Question Assumendo che al tempo zero i nuclei di tipo $N_1$ siano in numero $N_0$ e quelli di tipo $N_2$ e $N_3$
siano assenti, calcolare il numero di nuclei dei tre tipi per qualsiasi tempo. In particolare, si chiede il 
rapporto  $N_3 / N_1$ dopo 1/4 di secondo.
\end{Exercise}
\begin{Answer}
Nel caso di 3 decadimenti in sequenza si ha:
\beqn
\dv{N_1}{t} &=& - \omega_1 N_1 \\
\dv{N_2}{t} &=& \omega_1 N_1 - \omega_2 N_2 \\
\dv{N_3}{t} &=& \omega_2 N_2 - \omega_3 N_3~.
\eeqn
Le soluzioni particolari del sistema per le condizioni iniziali $N_1(0) = N_0$, $N_k(0) = 0$ e $dN_k/dt(0) = 0$ per k=2,3 sono:
\beqn
N_1(t) &=& N_0 e^{ - \omega_1 t} \\
N_2(t) &=& N_0 \frac{\omega_1}{\omega_2 - \omega_1} (e^{ - \omega_1 t} - e^{ - \omega_2 t}) \\
N_3(t) &=& N_0 \omega_1 \omega_2 \left[ \frac{e^{ - \omega_1 t}}{(\omega_2 - \omega_1)(\omega_3 - \omega_1)}
+ \frac{e^{ - \omega_2 t}}{(\omega_3 - \omega_2)(\omega_1 - \omega_2)}
+ \frac{e^{ - \omega_3 t}}{(\omega_1 - \omega_3)(\omega_2 - \omega_3)} \right ] ~.
\eeqn
Nel caso in esame abbiamo $\omega_3 = 0$ e quindi:
\[
N_3(t) = N_0 \left[ 1 + \frac{e^{ - \omega_1 t}}{\omega_1/\omega_2 - 1} + 
  \frac{e^{ - \omega_2 t}}{\omega_2/\omega_1 - 1} \right]~.
\]
In particolare dopo 1/4 di secondo abbiamo:
\[
\frac{N_3}{N_1} = \frac{\left[ 1 + \frac{e^{ - \omega_1 t}}{\omega_1/\omega_2 - 1} + 
    \frac{e^{ - \omega_2 t}}{\omega_2/\omega_1 - 1} \right]}{e^{ - \omega_1 t}} \approx 10.9~.
\]
\end{Answer}

\begin{Exercise}
  Sebbene l'interazione debole sia universale, diversi processi
  regolati da essa possono avvenire con probabilit\`a largamente
  diverse, a causa di motivi cinematici. Ad esempio, nei decadimenti:
  $\pi^- \to \mu^-\bar \nu$ e $\pi^- \to e^-\bar \nu$, per la natura
  ``V--A'' (Vettoriale--Assiale) delle interazioni deboli, l'elemento
  di matrice della transizione, $M^2$ \`e proporzionale a $1-\beta^2$,
  dove $\beta$ \`e la velocit\`a del leptone carico.

  \Question Assumendo tale relazione per l'elemento di matrica,
  stimare il rapporto tra i tassi di decadimento dei due canali:
  \[
  r = \frac{\Gamma(\pi^- \to \mu^-\bar \nu)}{\Gamma(\pi^- \to e^-\bar \nu)}
  \]

  [Dati: $m_\mu = \SI{105}{MeV/c^2}$, $m_e = \SI{0.511}{MeV/c^2}$, $m_\pi = \SI{140}{MeV/c^2}$]
\end{Exercise}
\begin{Answer}
  Per le interazioni deboli la rate di decadimento \`e data da:
  \[
  \Gamma = 2\pi G_F^2\vert M \vert^2\dv{N}{E_0}
  \]
  dove $\dv{N}{E_0}$ \`e il numero di stati finali per unit\`a di
  intervallo di energia, $M$ \`e l'elemento di matrice della
  transizione (cio\`e l'ampiezza per il decadimento considerato, dallo
  stato iniziale a quello finale, e $G_F$ \`e la costante di Fermi
  dell'interazione debole.

  Lo spazio delle fasi \`e:
  \[
  \dv{N}{E_0} = C p^2 \dv{p}{E_0}
  \]
  dove $C$ \`e una costante, $p$ il momento del leptone carico
  ($\ell=e,\mu$) nel sistema di riferimento del pione (nella notazione
  usuale, sarebbe $p^*$, ma scriviamo $p$ per semplificare la
  notazione in seguito).  L'energia totale del sistema \`e, usando la
  conservazione dell'energia nel sistema di riferimento del centro di
  massa:
  \[
  E_0 = m_\pi = \sqrt{p_e^2 +m_\ell^2} + \sqrt{p_\nu^2+m^2_\nu}
  \]
  usando la conservazione dell'impulso, e chiamando $p \equiv
  p_\nu=p_e$, e assumendo zero la massa del neutrino $m_\nu=0$,
  \[
  E_0 = m_\pi = p + \sqrt{p^2+m^2_e}
  \]
  e quindi si ricava:
  \[
  p = \frac{E_0^2-m_\ell^2}{2E_0}
  \]
  nel decadimento a due corpi $E_0=m_\pi$, e quindi:
  \[
  p = \frac{m_\pi^2-m_\ell^2}{2m_\pi}
  \]
  dalla relazione in funzione di $E_0$ si pu\`o ricavare uno dei fattori dello spazio delle fasi derivando $p(E_0)$:
  \[
  \dv{p}{E_0} = \frac{E_0}{2} - \frac{m_\ell^2}{2E_0} = \frac{E_0^2-m_\ell^2}{2E_0^2}
  \]
  e di nuovo, considerando che $E_0=m_\pi$:
  \[
  \dv{p}{E_0} = \frac{m_\pi^2+m_\ell^2}{2m_\pi^2}
  \]

  Ci rimane di calcolare l'elemento di matrice, che abbiamo detto essere, in questo caso, $M^2 \approx 1-\beta$.
  Calcoliamo la velocit\`a dell'elettrone:
  \[
  \beta = \frac{p}{\sqrt{p^2+m_\ell^2}} = \frac{p}{m_\pi-p}
  \]
  dove abbiamo usato ancora la conservazione dell'energia precedente. Per semplificare i passaggi algebrici, calcoliamo $1/\beta$:
  \[
  \frac{1}{\beta} = \frac{m_\pi}{p}-1 = \frac{2m_\pi^2}{m_\pi^2-m_\ell^2} - 1 = \frac{m_\pi^2+m_\ell^2}{m_\pi^2-m_\ell^2}
  \]
  da cui si ricava quello che ci serve:
  \[
  1-\beta = 1-\frac{m_\pi^2-m_\ell^2}{m_\pi^2+m_\ell^2}  = \frac{2m_\ell^2}{m_\pi^2+m_\ell^2}.
  \]
  Quindi possiamo finalmente calcolare il tasso di decadimento come proporzionale a:
  \[
  \Gamma \propto (1-\beta)p^2\dv{p}{E_0} = \left(\frac{2m_\ell^2}{m_\pi^2+m_\ell^2}\right) \left( \frac{m_\pi^2-m_\ell^2}{2m_\pi} \right)^2 \left( \frac{m_\pi^2+m_\ell^2}{2m_\pi^2} \right)
  \]
  e quindi:
  \[
  \Gamma \propto \frac{1}{4}\left(\frac{m_\ell}{m_\pi}\right)^2  \left( \frac{m_\pi^2-m_\ell^2}{m_\pi} \right)^2.
  \]
  Quindi il rapporto tra i tassi di decadimento \`e:
  \[
  r = \frac{\Gamma(\pi^- \to \mu^-\bar \nu)}{\Gamma(\pi^- \to e^-\bar \nu)} = \frac{m_\mu^2(m_\pi^2-m_\mu^2)^2}{m_e^2(m_\pi^2-m_e^2)^2} = \SI{8.13e3}{}
  \]
  Quindi, nonostante lo spazio delle fasi pi\`u grande nel decadimento
  del pione in elettrone, rispetto a quello del muone, il primo \`e
  largamente soppresso dal fattore $(m_\mu/m_e)^2 \approx (200)^2 =
  40000$ dovuto alla natura particolare dell'interazione debole
  (V--A).
\end{Answer}


\end{document}
