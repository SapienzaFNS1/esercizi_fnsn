%%%%%%%%%%%%%%%%%%%%%%%%
%%% ESERCITAZIONE 4
%%% FATTA IL 19/03/2021
%%%%%%%%%%%%%%%%%%%%%%%%

\begin{Exercise}[title={Unit\`a di misura}]
Usando il fatto che $\hbar c = \SI{197.3}{MeV fm}$, si dimostri che in un sistema di unità di misura in cui $\hbar=c=1$ vale:
\Question $\SI{1}{GeV^{-2}}=\SI{0.389}{mb}$
\Question $\SI{1}{m}=\SI{5.068e15}{GeV^{-1}}$
\Question $\SI{1}{s}=\SI{1.5e24}{GeV^{-1}}$
\end{Exercise}
Ricordiamo che $\SI{1}{b} = \SI{1e-28}{m^2}$ e che 
\begin{equation*}
    [\hbar c]=[\si{J s} \si{m/s}] = [E][L].
\end{equation*}
\begin{Answer}
L'idea è di capire per quale potenza di $\hbar c$ e $c$ va moltiplicato il termine a sinistra di ciascuna equazione, per ottenere il termine di destra. 

Per cui:
\begin{itemize}
    \item $[ \SI{1}{GeV^{-2}} ] [\hbar c]^\alpha = [E]^{-2} [E] ^\alpha [L]^\alpha = [\SI{0.389}{mb}] = [L]^2$, da cui segue $\alpha=2$ e $\SI{1}{GeV^{-2}}(\hbar c)^2=\frac{\SI{197.3}{MeV fm}}{\SI{1}{GeV}} = \left(\frac{\SI{0.1973e-15}{GeV m}}{\SI{1}{GeV}}\right)^2=\SI{0.389}{mb}$;

    \item $[ \SI{1}{m} ] [\hbar c]^\alpha = [L] [E] ^\alpha [L]^\alpha = [\SI{5.068e15}{GeV^{-1}}] = [E]^{-1}$, da cui segue $\alpha=-1$ e $\SI{1}{m}(\hbar c)^{-1}=\frac{\SI{1}{m}}{\SI{197.3}{MeV fm}} = \frac{\SI{1}{m}}{\SI{0.1973e-15}{GeV m}}=\SI{5.068e15}{GeV^{-1}}$;
    
    \item $[ \SI{1}{s} ] [\hbar c]^\alpha[c]^\beta  = [T] [E] ^\alpha [L]^\alpha [L]^\beta [T]^{-\beta} = [T] [E]^\alpha [L]^{\alpha+\beta} [T]^{-\beta} = [\SI{1.5e24}{GeV^{-1}}] = [E]^{-1}$, da cui segue $\alpha=-1, \beta=1$ e $\SI{1}{s}(\hbar c)^{-1}c=\frac{\SI{1}{s}}{\SI{197.3}{MeV fm}}\SI{299 792 458}{m/s} 
    = \frac{\SI{299 792 458}{m}}{\SI{0.1973e-15}{GeV m}} = \SI{1.5e24}{GeV^{-1}}$.
\end{itemize}
\end{Answer}


\begin{Exercise}[title={Energia di soglia di una reazione}]
  Nell’urto protone-nucleo calcolare l’energia cinetica di soglia
  minima e massima per la reazione protone su protone di un nucleo di
  rame:
  \begin{equation*}
    p_1 + p_2 \to p + p + \pi^+ \pi^-
  \end{equation*}
  sapendo che il moto di Fermi del protone nel nucleo-bersaglio ha un
  impulso medio di $p_F = \SI{0.240}{GeV/c}$ e che la massa del pione
  carico \`e: $m(\pi^\pm =\SI{0.140}{GeV/c^2}$ e la massa del protone
  \`e $m_p= \SI{0.938}{GeV/c^2}$.
\end{Exercise}

\begin{Answer}
L'impulso di Fermi regola il moto dei nucleoni (protoni e neutroni)
all'interno dei nuclei. Se lo trascuriamo l’energia cinetica di soglia
per la reazione è pari a:
\begin{equation*}
K_1 = \frac{(2m_p + 2m_\pi)^2-(2m_p)^2}{2m_p} = 4m_\pi + \frac{2m_\pi^2}{m_p} = \SI{0.602}{GeV}
\end{equation*}
L'energia totale \`e la somma di energia cinetica e della massa della particella prodotta:
\begin{equation*}
E_1 = K_1 + m_p = \SI{1.540}{GeV}
\end{equation*}

Considerando invece il moto del protone legato nel nucleo di rame,
moto diretto casualmente rispetto alla direzione del protone
incidente, si ha l’energia di soglia minima (massima) quando l’impulso
di Fermi del protone del nucleo è antiparallelo (parallelo) alla
direzione del protone incidente.

Conviene calcolare la massa invariante nel sistema del laboratorio:
\begin{equation*}
  \sqrt{s} = \sqrt{(E_1+E_2)^2-\abs{\vec p_1+\vec p_2}^2} = \sqrt{(E_1+E_2)^2-(p_1+p_F)^2} =
  \sqrt{2m_p^2 + 2E_1E_2 \pm 2p_1p_F}
\end{equation*}
l'energia del protone nel nucleo \`e $E_2 = \sqrt{m_p^2+m_F^2}=\SI{968}{MeV}$.
L'energia di soglia nel centro di massa \`e quando tutte le particelle dello stato finale sono ferme. Quindi \`e:
\begin{equation*}
\sqrt{s}= 2m_p+2m_\pi
\end{equation*}
Usando l'uguaglianza della massa invariante nello stato iniziale e finale:
\begin{eqnarray*}
  E_1E_2\pm p_1p_F = 2(m_p+m_\pi)^2-m_p^2 \\
  E_1E_2 - 2(m_p+m_\pi)^2 + m_p^2 = \pm p_F\sqrt{E_1^2-m_p^2}
\end{eqnarray*}
La parte $2(m_p+m_\pi)^2 + m_p^2$ \`e una costante, e vale $A=\SI{1.444}{GeV^2}$. Sostituendola:
\begin{eqnarray*}
  E_1E_2 - A =  \pm p_F\sqrt{E_1^2-m_p^2} \\
  p_F^2E_1^2 - p_F^2m_p^2 = E_1^2E_2^2 + A^2-2AE_1E_2 \\
  E_1^2(E_2^2-p_F^2) - 2AE_2E_1 + p_F^2m_p^2 + A^2=0
\end{eqnarray*}

e usando $E_2^2-p_F^2=m_p^2$ si trova:
\begin{eqnarray*}
  E_1^2m_p^2 - 2AE_1E_2 + p_F^2m_p^2+A^2 = 0 \\
  E_1^2-\frac{2AE_2}{m_p^2}E_1 + \frac{A^2}{m_p^2}+p_F^2=0
\end{eqnarray*}
Questa \`e un'equazione di secondo grado che ha soluzioni:
\begin{equation*}
  E_1 = \frac{AE_2}{m_p^2} \pm \sqrt{\left(\frac{AE_2}{m_p^2}\right)^2-\frac{A^2}{m_p^2}-p_F^2}
\end{equation*}  
Usando $E_2=\SI{968}{MeV}$ si ricavano i due valori di energia di soglia massimo e  minimo:
\begin{eqnarray*}
E_1^{max} \approx \SI{1.3}{GeV} \\
E_1^{min} \approx \SI{1.9}{GeV}
\end{eqnarray*}
Da cui si ricavano anche le energie cinetiche minime e massime:
\begin{eqnarray*}
K_1^{max} = E_1^{max} - m_p \approx \SI{1}{GeV} \\
K_1^{min} = E_1^{min} - m_p \approx \SI{0.3}{GeV} \\
\end{eqnarray*}

\end{Answer}



\begin{Exercise}[title={Energia di soglia di una reazione su bersaglio e collisore}]
  Calcolare l’energia di soglia per la reazione:
  \beq
  e^+e^- \to \mu^+\mu^-
  \eeq
  \Question su bersaglio fisso
  \Question in collisioni $e^+e^-$ con fasci di pari energia
\end{Exercise}
\begin{Answer}
  
Per la reazione su bersaglio fisso.
Come di frequente, usiamo due sistemi di riferimento diversi. Nello stato iniziale, considerando i quadrimpulsi nel laboratorio la massa invariante \`e:
\beq
\sqrt{s} = \sqrt{2m_e^2 + 2 E_1m_e}
\eeq
Nello stato finale, alla soglia (muoni fermi) si ha:
\beq
\sqrt{s} = 2m_\mu
\eeq
Uguagliando le due espressioni per la massa invariante si ottiene $E_1 = \SI{44}{GeV}$ alla soglia.

Per la reazione in collisioni $e^+e^-$ con fasci di pari energia, il sistema del centro di massa coincide con quello del laboratorio.
Nello stato iniziale:
\beq
\sqrt{s} = 2E_e
\eeq
Nello stato finale, alla soglia (ancora una volta, muoni fermi) si ha:
\beq
\sqrt{s} = 2m_\mu
\eeq
Uguagliando le due espressioni per la massa invariante si ottiene $E_e = m_\mu = \SI{0.106}{GeV}$ alla soglia.
Notare la grandissima differenza di energia necessaria per il fascio di elettroni nel primo e nel secondo caso.

\end{Answer}

\begin{Exercise}[title={Energia di soglia e decadimento in due corpi}]
  Si consideri la collisione frontale tra un fascio di protoni ed uno
  di elettroni, di pari impulso p nel sistema di riferimento del
  laboratorio, che produce la reazione:
  \beq
  e^- + p \to \Lambda + \nu_e
  \eeq
  \Question Determinare l’energia dell’elettrone quando la reazione e`
  prodotta a soglia con $\Lambda \to p \pi^-$.
  \Question Determinare l’impulso del protone e del pione, prodotti dal decadimento della
  $\Lambda$, nel sistema di riferimento in cui la $\Lambda$ \`e in quiete
  [$m_e = \SI{0.511}{MeV/c^2}$, $m_p = \SI{938.3}{MeV/c^2}$, $m_\Lambda = \SI{1115.7}{MeV/c^22}$, $m_\pi = {139.6}{MeV/c^2}$]
\end{Exercise}
\begin{Answer}
  Il sistema del laboratorio coincide col sistema del centro di massa. L’impulso di soglia si ricava quindi imponendo:
  \beq
  \sqrt{s}=\sqrt{p^2+m_e^2}+\sqrt{p^2+m_p^2}=m_\Lambda
  \eeq
  Elevando al quadrato si ha:
  \beq
  p^2+m_e^2+p^2+m_p^2+2\sqrt{(p^2+m_e^2)(p^2+m_p^2)}=m_\Lambda^2 \\
  \eeq
  Conviene portare al secondo membro la radice:
  \beq
  2p^2+m_e^2+m_p^2 - m_\Lambda^2 = 2\sqrt{(p^2+m_e^2)(p^2+m_p^2)}
  \eeq
  e poi elevando al quadrato:
  \beq
  4p^4+m_e^4+m_p^4+m_\Lambda^4+4p^2m_e^2+4p^2m_p^2-4p^2m_\Lambda^2 - 2m_e^2m_p^2-2m_e^2m_\Lambda^2-2m_p^2m_\Lambda^2 =
  4p^4+4p^2m_p^2+4p^2m_e^2+4m_e^2m_p^2
  \eeq
  si ottiene:
  \beq
  p^2=\frac{m_e^4+m_p^4+m_\Lambda^4-2me^2m_p^2-2m_e^2m\Lambda^2-2m_p^2m_\Lambda^2}{4m_\Lambda^2}
  \eeq
  e quindi il valore di $p=\SI{163}{MeV/c}$, da cui l'energia $E_e=pc=\SI{163}{MeV}$.

  Nel caso del successivo decadimento in due corpi $\Lambda\to p\pi^-$, si ottiene l'energia del protone, nel sistema di
  riferimento del centro di massa, usando la formula:
  \beq
  E_p^* = \frac{m_\Lambda^2+m_p^2-m_\pi^2}{2m_\Lambda} = \SI{944}{MeV}
  \eeq
  L'impulso del protone \`e quindi dato da:
  \beq
  p_p^* = \sqrt{{E_p^*}^2-m_p^2}=\SI{100}{MeV/c}
  \eeq
  Siccome nel centro di massa, per definizione, la somma vettoriale
  degli impulsi \`e nulla, l'impulso del pione carico bilancia quello
  del protone, emesso in direzione opposta, e quindi $p_\pi^*=p_p^* =\SI{100}{MeV/c}$.
\end{Answer}

\begin{Exercise}[title={Diffusione elastica di un fotone su un bersaglio}]
Chiamiamo elastico un urto (``scattering'') in cui le particelle dello
stato iniziale e dello stato finale sono le stesse. Si consideri un
urto elastico fra una particella di massa nulla e una particella di
massa $m$ (\emph{bersaglio}) che si trova a riposo nel sistema di
riferimento del laboratorio: qual è la massima energia trasferita
dalla particella incidente al bersaglio? \emph{Suggerimento: si lavori
nel sistema di riferimento del laboratorio, e si espliciti il prodotto
scalare fra gli impulsi spaziali della particella di massa nulla prima
e dopo l'urto in funzione dell'angolo, sempre nel sistema di
riferimento del laboratorio, fra la direzione iniziale e finale della
particella incidente.}

Se la particella incidente è un fotone e il bersaglio è un elettrone atomico a riposo, di quanto varia la lunghezza d'onda del fotone fra prima e dopo l'urto?
\end{Exercise}

\begin{Answer}
Per scattering elastico intendiamo un processo in cui le particelle dello stato iniziale sono le stesse di quelle dello stato finale.

Denotiamo con $\quadriv{k}$ e $\quadriv{P}$ i quadrimpulsi della particella incidente e del bersaglio prima dell'urto, e indichiamo con l'apice le stesse quantità dopo l'urto: il problema ci dice che
\begin{align*}
    \quadriv{k} = (E, \vett{k}),\\
    \quadriv{k'} = (E', \vett{k'}),\\
    \quadriv{P} = (mc, \vett{0}).\\
\end{align*}
Partiamo dalla conservazione del quadrimpulso durante l'urto, isoliamo la quantità che non misuriamo direttamente -- cioè il quadrimpulso del bersaglio dopo l'urto, $\quadriv{P'}$ -- ed eleviamo al quadrato:
\begin{align*}
    \quadriv{k} + \quadriv{P} = \quadriv{k'} + \quadriv{P'},\\
    \quadriv{P'} = \quadriv{k} + \quadriv{P} - \quadriv{k'},\\
    m^2c^2 = 0 + m^2c^2 + 0 + 2Em-2(EE'-\vett{k}\cdot\vett{k'})-2mE',
\end{align*}
e se indichiamo con $\theta'$ l'angolo -- nel riferimento del laboratorio -- fra la direzione iniziale e finale della particella incidente, e usiamo il fatto che $|\vett{k}|c = E$ e $|\vett{k'}|c = E'$,
\begin{align*}
    0= 2mc^2(E-E')-2(EE'-EE'\cos\theta'),\\
    mc^2(E'-E) = -EE'(1-\cos\theta'),\\
    E'(mc^2 + E(1-\cos\theta')) = mc^2E,\\
    E' = \frac{E}{1 + \frac{E}{mc^2}(1-\cos\theta')}.
\end{align*}
Il bersaglio rinculerà di una energia $E-E'$, massima per $\theta=\pi$. Il valore massimo di quest'energia di rinculo,
\begin{equation*}
    E-\frac{E}{1+2\frac{E}{mc^2}} = E\frac{2E/mc^2}{1+2E/mc^2},
\end{equation*}
prende il nome -- nel caso dello scattering Compton, in cui la particella incidente è un fotone e il bersaglio è un elettrone atomico -- di \emph{picco Compton}.

Cosa cambia fra un fotone di energia $E$ ed uno di energia $E'$? Dalla meccanica quantistica,
\begin{equation*}
    E = h\nu = \frac{hc}{\lambda},
\end{equation*}
cioè cambia la lunghezza d'onda del fotone:
\begin{align*}
    E' = \frac{hc}{\lambda'} = \frac{\frac{hc}{\lambda}}{1+\frac{\frac{hc}{\lambda}}{mc^2}(1-\cos\theta')},\\
    \frac{1}{\lambda'} = \frac{\frac{1}{\lambda}}{1+\frac{\frac{hc}{\lambda}}{mc^2}(1-\cos\theta')},\\
    \lambda' = {\lambda}\left(1+\frac{hc}{\lambda mc^2}(1-\cos\theta')\right),\\
    \lambda' = \lambda + \frac{h}{m c}(1-\cos\theta')\equiv \lambda + \lambda_c(1-\cos\theta').
\end{align*}
dove abbiamo definito la \emph{lunghezza d'onda Compton} dell'elettrone, $\lambda_c$, che rappresenta la scala di lunghezza sotto la quale gli effetti della meccanica quantistica relativistica divengono importanti.
\end{Answer}

\begin{Exercise}[title={Scattering Rutherford}]
Un fascio di particelle $\alpha$ di \SI{100}{MeV} di energia e \SI{0.32}{nA} di corrente\footnote{Per una spiegazione breve su come (e perché) si misura la corrente di un fascio di particelle, vedi \url{https://www.lhc-closer.es/taking_a_closer_look_at_lhc/0.beam_current}. Una trattazione più completa è data ad esempio da \url{https://cds.cern.ch/record/1213275/files/p141.pdf}.} collide contro un bersaglio fisso di alluminio, spesso \SI{1}{cm}. Una sperimentatrice prende un rivelatore di $\SI{1}{cm}\times\SI{1}{cm}$ di superficie, e lo posiziona ad un angolo di \ang{30} rispetto al fascio di particelle, a \SI{1}{m} di distanza dal bersaglio. Quante particelle $\alpha$ incideranno sul rivelatore ogni secondo?
\end{Exercise}
\begin{Answer}
L'alluminio ha una densità di \SI{2.7}{g/cm^3}, numero atomico $13$ e massa atomica \SI{27}{u}.

Poiché le particelle $\alpha$ sono nuclei di elio, hanno carica $2e$ e la corrente di \SI{0.32}{nA} corrisponde a un miliardo di particelle incidenti al secondo,
\[
\dv{N_i}{t}=\frac{\SI{0.32}{nC/s}}{2\times\SI{1.6e-19}{C}} = \SI{1e9}{s^{-1}}.
\]

Il rivelatore vede un angolo solido di
\[
\Delta\Omega\equiv \frac{\text{superficie}}{\text{raggio}}^2 = \frac{\SI{1}{cm^2}}{(\SI{1}{m})^2} = \SI{1e-4}{sr}
\]

Si tratta di uno scattering alla Rutherford, per cui la sezione d'urto per unità di angolo solido rilevata ad un certo angolo $\theta$ vale
\[
\dv{\sigma}{\Omega} = \left(\frac{z_{\alpha}z_{Al}e^2}{4\pi\epsilon_04E}\frac{1}{\sin^2(\theta/2)}\right)^2,
\]
pari a
\begin{equation}\begin{split}
\dv{\sigma}{\Omega} \approx \left(\frac{2\times13\times4\times e\times\SI{1.6e-19}{C}}{4\pi\times\SI{8.9e-12}{F/m}\times4\times\SI{100e6}{eV}}\frac{1}{\sin^2(\pi/\ang{180}\times\ang{30}/2)}\right)^2\\
\approx \SI{2e-30}{m^2/sr}=\SI{20}{mb/sr},
\end{split}\end{equation}
e il numero di particelle visto dal rivelatore vale, se indichiamo con $n_{Al}=\rho_{Al}\frac{N_A}{A_{Al}}$ la densità numero di atomi di alluminio, e con $d$ lo spessore del rivelatore,
\begin{equation*}
\begin{split}
&\dv{N_\text{rivelate}}{t}=\Delta\Omega\dv{\sigma}{\Omega}n_{Al}d \dv{N_\text{i}}{t}\\ 
&\approx\SI{1e-4}{sr}\times \SI{2e-30}{m^2/sr} \times \SI{1e4}{cm^2/m^2}  \times \SI{2.7}{g/cm^3} \frac{\SI{6e23}{mol^{-1}}}{\SI{27}{g/mol}}\\&= \SI{120}{Hz}.
\end{split}
\end{equation*}
\end{Answer}
