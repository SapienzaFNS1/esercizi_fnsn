\begin{Exercise}[title={Densit\`a nucleare e unit\`a di misura}]
  Si stimi la densit\`a nucleare in $\SI{}{g cm^3}$, approssimando $m_p = m_n = \SI{938.3}{MeV/c^2}$.
\end{Exercise}
\begin{Answer}
Un elettronVolt (eV) \`e l'energia cinetica acquistata da una
particella di carica elementare $e =\SI{1.602e-19}{C}$ che passa
attraverso la differenza di potenziale di un Volt, per cui:
\[
\SI{1}{eV} = \SI{1.602e-19}{J}.
\]
Il fattore di conversione tra massa (che in relativit\`a \`e equivalente a un'energia) \`e dato da:
\[
E(\SI{1}{kg} = \SI{1}{kg c^2} = \SI{9e16}{kg(m/s)^2} = \SI{9e16}{J}
\]
e poich\'e $\SI{1}{J}=1/\SI{1.6e-19}{eV}$ si ha:
\[
\SI{1}{kg c^2} = \frac{9e16}{1.6e-19}\SI{}{eV} = \SI{5.6e35}{eV} 
\]
da cui:
\[
\SI{1}{kg} = \SI{5.6e35}{eV/c^2} 
\]
Quindi la massa del protone, espressa in kg, vale:
\[
m_p = \SI{9.383e8}{eV/c^2} = \frac{\SI{9.383e8}{}}{\SI{5.6e35}{}}\SI{}{kg} = \SI{1.67e-27}{kg}
\]

Il testo dice di considerare $m_n \approx m_p = \SI{1.67e-24}{kg}$. Quindi usando il raggio del nucleo:
\[
R = R_0 \cdot A^{1/3}
\]
con $R_0=\SI{1.2}{fm}$, che \`e valido per grandi valori di $A$, si ottiene:
\beqn
\rho &=& \frac{M}{V} = \frac{A\cdot m_p}{4/3 \pi R_0^3 A} = \frac{3 m_p}{4\pi R_0^3} \\
&=& \frac{3 \times \SI{1.67e-27}{g}}{12.56 \times (\SI{1.2e-13}{cm})^3} \approx \SI{2.3e14}{g/cm^3}
\eeqn
\end{Answer}

\begin{Exercise}[title={Termine coulombiano della formula di Weiszacker}]
Considerando che l'energia elettrostatica di una carica $Q$
uniformemente distribuita su una sfera di raggio $R$ \`e uguale a
$\frac{3}{5}\cdot\frac{Q^2}{4\pi\epsilon_0 R}$, stimare il termine
coulombiano della formula semi-empirica delle masse dei nuclei.
\end{Exercise}
\begin{Answer}
La carica del nucleo \`e $Q=Ze$, mentre il raggio del nucleo si pu\`o stimare con:
\[
R = R_0 \cdot A^{1/3}
\]
dove $R_0=\SI{1.2}{fm}$, che \`e valido per grandi valori di $A$.
Il termine coulombiano della formula di Weiszacker \`e quello proporzionale a $1/R$, quindi:
\[
\frac{3Z^2e^2}{20\pi \epsilon_0 R_0A^{1/3}} = a_C \cdot \frac{Z^2}{A^{1/3}}
\]
da cui si ricava il coefficiente $a_C$ del termine coulombiano, fattorizzandolo in modo opportuno da
evidenziare delle costanti di cui sappiamo il valore:
\[
a_C = \frac{3}{5}\times\frac{e^2}{4\pi\epsilon_0}\times\frac{1}{R_0} =
0.6\times k e^2 \times \frac{1}{\SI{1.2}{fm}} 
\]
usando il valore della costante di Coulomb:
\[
k e^2 = \frac{\SI{197}{MeV \cdot fm}}{137}=\SI{1.44}{MeV \cdot fm}
\]
si ottiene:
\[
a_C \approx \SI{0.7}{MeV}
\]
che \`e una buona approssimazione per tale costante.
\end{Answer}

\begin{Exercise}[title={Termine coulombiano della formula di Weiszacker}]
  Gli stati stabili di \ce{^{13}_{6}C} e \ce{^{13}_7N} appartengono
  allo stesso doppietto di isospin, e la loro differenza di massa \`e
  dovuta principalmente dalla diversa energia coulombiana, e in modo
  sottodominante dalla differenza di massa tra neutrone e protone.

  \Question Considerando entrambe le cause di differenza di massa, si
  stimi la differenza di massa dei due nuclei.
\end{Exercise}
\begin{Answer}
  Come nell'esercizio precedente, approssimando la carica elettrostatica come uniformemente
  distribuita in delle sfere di raggio $R=R_0 A^{1/3}$, con $R_0 = \SI{1.2}{fm}$, la loro
  energia elettrostatica \`e, per una carica $Q$:
  \[
  E_C = \frac{3}{5}\times\frac{Q^2}{4\pi\epsilon_0 R}
  \]
  Poich\'e il \ce{^{13}_7N} ha un protone in pi\`u rispetto al \ce{^{13}_{6}C}, esso avr\`a un'energia coulombiana maggiore.
  Per\`o bisogna considerare che il neutrone ha minor massa rispetto al protone:
  \[
  M_n - M_p - m_e = \SI{0.782}{MeV/c^2}
  \]
  Quindi la differenza di massa tra i due nuclei \`e data da:
  \beqn
      [M(\ce{^{13}_7N}) - M(\ce{^{13}_{6}C})] &=& \frac{3}{5R\times 4\pi\epsilon_0}(Q^2_N-Q^2_C) - [M_n - M_p - m_e] \\
      &=& \frac{3}{5R}\left(\frac{e^2}{4\pi\epsilon_0}\right)(7^2-6^2) - \SI{0.782}{MeV} \\
      &=& 0.6 \times \SI{1.44}{MeV\cdot fm} \times \frac{49-36}{\SI{1.2}{fm}\times 13^{1/3}} - \SI{0.782}{MeV} \\
      &\approx& \SI{2.62}{MeV}
  \eeqn
\end{Answer}

\begin{Exercise}[title={Applicazione della legge di Geiger-Nuttal al decadimento $\alpha$}]
  La relazione di Geiger-Nuttal lega in modo semplice la costante di decadimento ($\lambda$) della
  radioattivit\`a naturale $\alpha$ e l'energia della particella $\alpha$ ($E_\alpha$) emessa.

  \Question Usando tale legge, si stimi l'ordine di grandezza per il
  tempo di dimezzamento del \ce{^{210}Po}, che decade emettendo
  $E_\alpha=\SI{5.3}{MeV}$, sapendo che il \ce{^{214}Po} ha un tempo
  di dimezzamento di $\SI{1.6e-4}{s}$, ed emette una particella
  $\alpha$ con $E_\alpha=\SI{7.7}{MeV}$.
\end{Exercise}
\begin{Answer}
  Per radionuclidi della stessa serie che decadono emettendo particelle $\alpha$, la legge di Geiger--Nuttal lega in
  modo lineare $\ln \lambda$ e $1/\sqrt{E_\alpha}$. Per il \ce{_{84}Po} si ha:
  \[
  \ln \lambda \approx a - \frac{Z-2}{E_\alpha}b
  \]
  Per i due isotopi del \ce{_{84}Po}, che hanno lo stesso $Z=84$, si ha:
  \beqn
  \ln \lambda(\ce{^{210}Po}) &\approx& a - \frac{Z-2}{E_\alpha(\ce{^{210}Po})}b \\
  \ln \lambda(\ce{^{214}Po}) &\approx& a - \frac{Z-2}{E_\alpha(\ce{^{214}Po})}b
  \eeqn
  poich\'e le costanti $a$ e $b$ sono debolmente dipendenti per gli
  isotopi, sottraendo dalla seconda la prima equazione si ottiene:
  \[
  \ln \lambda(\ce{^{214}Po}) - \ln \lambda(\ce{^{210}Po}) = b (Z-2)\left[\frac{1}{E_\alpha(\ce{^{210}Po})} - \frac{1}{E_\alpha(\ce{^{210}Po})}\right]
  \]
  ovvero:
  \[
  \ln \frac{\lambda(\ce{^{214}Po})}{\lambda(\ce{^{210}Po})} = b (Z-2)\left[\frac{1}{\sqrt{E_\alpha(\ce{^{210}Po})}} - \frac{1}{\sqrt{E_\alpha(\ce{^{210}Po})}}\right]
  \]
  Sappiamo che la costante $b$ \`e pari a:
  \[
  b \equiv \frac{e^2\sqrt{2m_\alpha}}{2\hbar\epsilon_0} = \frac{2\pi e^2\sqrt{2m_\alpha}}{4\pi\hbar\epsilon_0}
  =2\pi \frac{ke^2\sqrt{2m_\alpha c^2}}{\hbar c}= 6.28 \frac{\SI{1.44}{MeV\cdot fm} \cdot \sqrt{2\cdot\SI{3727}{MeV/c^2}}}{\SI{197}{MeV\cdot fm}} \approx \SI{4}{MeV^{1/2}}
  \]
  Per avere una stima del rapporto in ordini di grandezza tra i due tempi di dimezzamento consideriamo che:
  \[
  \ln \frac{\lambda(\ce{^{214}Po})}{\lambda(\ce{^{210}Po})} = \ln \frac{t_{1/2}(\ce{^{210}Po})}{t_{1/2}(\ce{^{214}Po})}
  \]
  e quindi
  \beqn
  \log_{10}\frac{t_{1/2}(\ce{^{210}Po})}{t_{1/2}(\ce{^{214}Po})} &=& \frac{1}{\ln(10)} \ln \frac{t_{1/2}(\ce{^{210}Po})}{t_{1/2}(\ce{^{214}Po})} \\
  &\approx&0.434 \times  \SI{4}{MeV^{1/2}} \times(84-2)\times
  \left[\frac{1}{\sqrt{\SI{5.3}{MeV}}} - \frac{1}{\sqrt{\SI{7.7}{MeV}}}\right] \approx 10.5
  \eeqn
  Le vite medie dei due nuclidi differiscono di ben 10 ordini di grandezza. Conoscendo quindi il tempo di dimezzamento
  del \ce{^{214}Po} si trova il valore numerico della vita media del \ce{^{210}Po}:

  \[
  t_{1/2}(\ce{^{210}Po}) \approx 10^{10.5} \cdot t_{1/2}(\ce{^{214}Po}) \approx 10^{10} \cdot 10^{1/2} \cdot \SI{1.6e6}{s} = \SI{58.6}{giorni}.
  \]

  Il valore sperimentale per il tempo di dimezzamento del
  \ce{^{210}Po} \`e circa 140 giorni, che \`e dello stesso
  ordine di grandezza del valore stimato con la legge di
  Geiger-Nuttal.
\end{Answer}

\begin{Exercise}[title={Decadimenti in sequenza}]
 Si consideri la seguente sequenza di decadimento:
\begin{itemize}
\item $N_1 \to N_2$, con costante di decadimento $\omega_1 = 10~s^{-1}$;
\item $N_2 \to N_3$, con costante di decadimento $\omega_2 = 50~s^{-1}$;
\item $N_3$ \`e stabile.
\end{itemize}

\Question Assumendo che al tempo zero i nuclei di tipo $N_1$ siano in numero $N_0$ e quelli di tipo $N_2$ e $N_3$
siano assenti, calcolare il numero di nuclei dei tre tipi per qualsiasi tempo. In particolare, si chiede il 
rapporto  $N_3 / N_1$ dopo 1/4 di secondo.
\end{Exercise}
\begin{Answer}
Nel caso di 3 decadimenti in sequenza si ha:
\beqn
\dv{N_1}{t} &=& - \omega_1 N_1 \\
\dv{N_2}{t} &=& \omega_1 N_1 - \omega_2 N_2 \\
\dv{N_3}{t} &=& \omega_2 N_2 - \omega_3 N_3~.
\eeqn
Le soluzioni particolari del sistema per le condizioni iniziali $N_1(0) = N_0$, $N_k(0) = 0$ e $dN_k/dt(0) = 0$ per k=2,3 sono:
\beqn
N_1(t) &=& N_0 e^{ - \omega_1 t} \\
N_2(t) &=& N_0 \frac{\omega_1}{\omega_2 - \omega_1} (e^{ - \omega_1 t} - e^{ - \omega_2 t}) \\
N_3(t) &=& N_0 \omega_1 \omega_2 \left[ \frac{e^{ - \omega_1 t}}{(\omega_2 - \omega_1)(\omega_3 - \omega_1)}
+ \frac{e^{ - \omega_2 t}}{(\omega_3 - \omega_2)(\omega_1 - \omega_2)}
+ \frac{e^{ - \omega_3 t}}{(\omega_1 - \omega_3)(\omega_2 - \omega_3)} \right ] ~.
\eeqn
Nel caso in esame abbiamo $\omega_3 = 0$ e quindi:
\[
N_3(t) = N_0 \left[ 1 + \frac{e^{ - \omega_1 t}}{\omega_1/\omega_2 - 1} + 
  \frac{e^{ - \omega_2 t}}{\omega_2/\omega_1 - 1} \right]~.
\]
In particolare dopo 1/4 di secondo abbiamo:
\[
\frac{N_3}{N_1} = \frac{\left[ 1 + \frac{e^{ - \omega_1 t}}{\omega_1/\omega_2 - 1} + 
    \frac{e^{ - \omega_2 t}}{\omega_2/\omega_1 - 1} \right]}{e^{ - \omega_1 t}} \approx 10.9~.
\]
\end{Answer}
